\documentclass[11pt]{article}

    \usepackage[breakable]{tcolorbox}
    \usepackage{parskip} % Stop auto-indenting (to mimic markdown behaviour)
    

    % Basic figure setup, for now with no caption control since it's done
    % automatically by Pandoc (which extracts ![](path) syntax from Markdown).
    \usepackage{graphicx}
    % Keep aspect ratio if custom image width or height is specified
    \setkeys{Gin}{keepaspectratio}
    % Maintain compatibility with old templates. Remove in nbconvert 6.0
    \let\Oldincludegraphics\includegraphics
    % Ensure that by default, figures have no caption (until we provide a
    % proper Figure object with a Caption API and a way to capture that
    % in the conversion process - todo).
    \usepackage{caption}
    \DeclareCaptionFormat{nocaption}{}
    \captionsetup{format=nocaption,aboveskip=0pt,belowskip=0pt}

    \usepackage{float}
    \floatplacement{figure}{H} % forces figures to be placed at the correct location
    \usepackage{xcolor} % Allow colors to be defined
    \usepackage{enumerate} % Needed for markdown enumerations to work
    \usepackage{geometry} % Used to adjust the document margins
    \usepackage{amsmath} % Equations
    \usepackage{amssymb} % Equations
    \usepackage{textcomp} % defines textquotesingle
    % Hack from http://tex.stackexchange.com/a/47451/13684:
    \AtBeginDocument{%
        \def\PYZsq{\textquotesingle}% Upright quotes in Pygmentized code
    }
    \usepackage{upquote} % Upright quotes for verbatim code
    \usepackage{eurosym} % defines \euro

    \usepackage{iftex}
    \ifPDFTeX
        \usepackage[T1]{fontenc}
        \IfFileExists{alphabeta.sty}{
              \usepackage{alphabeta}
          }{
              \usepackage[mathletters]{ucs}
              \usepackage[utf8x]{inputenc}
          }
    \else
        \usepackage{fontspec}
        \usepackage{unicode-math}
    \fi

    \usepackage{fancyvrb} % verbatim replacement that allows latex
    \usepackage{grffile} % extends the file name processing of package graphics
                         % to support a larger range
    \makeatletter % fix for old versions of grffile with XeLaTeX
    \@ifpackagelater{grffile}{2019/11/01}
    {
      % Do nothing on new versions
    }
    {
      \def\Gread@@xetex#1{%
        \IfFileExists{"\Gin@base".bb}%
        {\Gread@eps{\Gin@base.bb}}%
        {\Gread@@xetex@aux#1}%
      }
    }
    \makeatother
    \usepackage[Export]{adjustbox} % Used to constrain images to a maximum size
    \adjustboxset{max size={0.9\linewidth}{0.9\paperheight}}

    % The hyperref package gives us a pdf with properly built
    % internal navigation ('pdf bookmarks' for the table of contents,
    % internal cross-reference links, web links for URLs, etc.)
    \usepackage{hyperref}
    % The default LaTeX title has an obnoxious amount of whitespace. By default,
    % titling removes some of it. It also provides customization options.
    \usepackage{titling}
    \usepackage{longtable} % longtable support required by pandoc >1.10
    \usepackage{booktabs}  % table support for pandoc > 1.12.2
    \usepackage{array}     % table support for pandoc >= 2.11.3
    \usepackage{calc}      % table minipage width calculation for pandoc >= 2.11.1
    \usepackage[inline]{enumitem} % IRkernel/repr support (it uses the enumerate* environment)
    \usepackage[normalem]{ulem} % ulem is needed to support strikethroughs (\sout)
                                % normalem makes italics be italics, not underlines
    \usepackage{soul}      % strikethrough (\st) support for pandoc >= 3.0.0
    \usepackage{mathrsfs}
    

    
    % Colors for the hyperref package
    \definecolor{urlcolor}{rgb}{0,.145,.698}
    \definecolor{linkcolor}{rgb}{.71,0.21,0.01}
    \definecolor{citecolor}{rgb}{.12,.54,.11}

    % ANSI colors
    \definecolor{ansi-black}{HTML}{3E424D}
    \definecolor{ansi-black-intense}{HTML}{282C36}
    \definecolor{ansi-red}{HTML}{E75C58}
    \definecolor{ansi-red-intense}{HTML}{B22B31}
    \definecolor{ansi-green}{HTML}{00A250}
    \definecolor{ansi-green-intense}{HTML}{007427}
    \definecolor{ansi-yellow}{HTML}{DDB62B}
    \definecolor{ansi-yellow-intense}{HTML}{B27D12}
    \definecolor{ansi-blue}{HTML}{208FFB}
    \definecolor{ansi-blue-intense}{HTML}{0065CA}
    \definecolor{ansi-magenta}{HTML}{D160C4}
    \definecolor{ansi-magenta-intense}{HTML}{A03196}
    \definecolor{ansi-cyan}{HTML}{60C6C8}
    \definecolor{ansi-cyan-intense}{HTML}{258F8F}
    \definecolor{ansi-white}{HTML}{C5C1B4}
    \definecolor{ansi-white-intense}{HTML}{A1A6B2}
    \definecolor{ansi-default-inverse-fg}{HTML}{FFFFFF}
    \definecolor{ansi-default-inverse-bg}{HTML}{000000}

    % common color for the border for error outputs.
    \definecolor{outerrorbackground}{HTML}{FFDFDF}

    % commands and environments needed by pandoc snippets
    % extracted from the output of `pandoc -s`
    \providecommand{\tightlist}{%
      \setlength{\itemsep}{0pt}\setlength{\parskip}{0pt}}
    \DefineVerbatimEnvironment{Highlighting}{Verbatim}{commandchars=\\\{\}}
    % Add ',fontsize=\small' for more characters per line
    \newenvironment{Shaded}{}{}
    \newcommand{\KeywordTok}[1]{\textcolor[rgb]{0.00,0.44,0.13}{\textbf{{#1}}}}
    \newcommand{\DataTypeTok}[1]{\textcolor[rgb]{0.56,0.13,0.00}{{#1}}}
    \newcommand{\DecValTok}[1]{\textcolor[rgb]{0.25,0.63,0.44}{{#1}}}
    \newcommand{\BaseNTok}[1]{\textcolor[rgb]{0.25,0.63,0.44}{{#1}}}
    \newcommand{\FloatTok}[1]{\textcolor[rgb]{0.25,0.63,0.44}{{#1}}}
    \newcommand{\CharTok}[1]{\textcolor[rgb]{0.25,0.44,0.63}{{#1}}}
    \newcommand{\StringTok}[1]{\textcolor[rgb]{0.25,0.44,0.63}{{#1}}}
    \newcommand{\CommentTok}[1]{\textcolor[rgb]{0.38,0.63,0.69}{\textit{{#1}}}}
    \newcommand{\OtherTok}[1]{\textcolor[rgb]{0.00,0.44,0.13}{{#1}}}
    \newcommand{\AlertTok}[1]{\textcolor[rgb]{1.00,0.00,0.00}{\textbf{{#1}}}}
    \newcommand{\FunctionTok}[1]{\textcolor[rgb]{0.02,0.16,0.49}{{#1}}}
    \newcommand{\RegionMarkerTok}[1]{{#1}}
    \newcommand{\ErrorTok}[1]{\textcolor[rgb]{1.00,0.00,0.00}{\textbf{{#1}}}}
    \newcommand{\NormalTok}[1]{{#1}}

    % Additional commands for more recent versions of Pandoc
    \newcommand{\ConstantTok}[1]{\textcolor[rgb]{0.53,0.00,0.00}{{#1}}}
    \newcommand{\SpecialCharTok}[1]{\textcolor[rgb]{0.25,0.44,0.63}{{#1}}}
    \newcommand{\VerbatimStringTok}[1]{\textcolor[rgb]{0.25,0.44,0.63}{{#1}}}
    \newcommand{\SpecialStringTok}[1]{\textcolor[rgb]{0.73,0.40,0.53}{{#1}}}
    \newcommand{\ImportTok}[1]{{#1}}
    \newcommand{\DocumentationTok}[1]{\textcolor[rgb]{0.73,0.13,0.13}{\textit{{#1}}}}
    \newcommand{\AnnotationTok}[1]{\textcolor[rgb]{0.38,0.63,0.69}{\textbf{\textit{{#1}}}}}
    \newcommand{\CommentVarTok}[1]{\textcolor[rgb]{0.38,0.63,0.69}{\textbf{\textit{{#1}}}}}
    \newcommand{\VariableTok}[1]{\textcolor[rgb]{0.10,0.09,0.49}{{#1}}}
    \newcommand{\ControlFlowTok}[1]{\textcolor[rgb]{0.00,0.44,0.13}{\textbf{{#1}}}}
    \newcommand{\OperatorTok}[1]{\textcolor[rgb]{0.40,0.40,0.40}{{#1}}}
    \newcommand{\BuiltInTok}[1]{{#1}}
    \newcommand{\ExtensionTok}[1]{{#1}}
    \newcommand{\PreprocessorTok}[1]{\textcolor[rgb]{0.74,0.48,0.00}{{#1}}}
    \newcommand{\AttributeTok}[1]{\textcolor[rgb]{0.49,0.56,0.16}{{#1}}}
    \newcommand{\InformationTok}[1]{\textcolor[rgb]{0.38,0.63,0.69}{\textbf{\textit{{#1}}}}}
    \newcommand{\WarningTok}[1]{\textcolor[rgb]{0.38,0.63,0.69}{\textbf{\textit{{#1}}}}}
    \makeatletter
    \newsavebox\pandoc@box
    \newcommand*\pandocbounded[1]{%
      \sbox\pandoc@box{#1}%
      % scaling factors for width and height
      \Gscale@div\@tempa\textheight{\dimexpr\ht\pandoc@box+\dp\pandoc@box\relax}%
      \Gscale@div\@tempb\linewidth{\wd\pandoc@box}%
      % select the smaller of both
      \ifdim\@tempb\p@<\@tempa\p@
        \let\@tempa\@tempb
      \fi
      % scaling accordingly (\@tempa < 1)
      \ifdim\@tempa\p@<\p@
        \scalebox{\@tempa}{\usebox\pandoc@box}%
      % scaling not needed, use as it is
      \else
        \usebox{\pandoc@box}%
      \fi
    }
    \makeatother

    % Define a nice break command that doesn't care if a line doesn't already
    % exist.
    \def\br{\hspace*{\fill} \\* }
    % Math Jax compatibility definitions
    \def\gt{>}
    \def\lt{<}
    \let\Oldtex\TeX
    \let\Oldlatex\LaTeX
    \renewcommand{\TeX}{\textrm{\Oldtex}}
    \renewcommand{\LaTeX}{\textrm{\Oldlatex}}
    % Document parameters
    % Document title
    \title{MonteCarloHedge}
    
    
    
    
    
    
    
% Pygments definitions
\makeatletter
\def\PY@reset{\let\PY@it=\relax \let\PY@bf=\relax%
    \let\PY@ul=\relax \let\PY@tc=\relax%
    \let\PY@bc=\relax \let\PY@ff=\relax}
\def\PY@tok#1{\csname PY@tok@#1\endcsname}
\def\PY@toks#1+{\ifx\relax#1\empty\else%
    \PY@tok{#1}\expandafter\PY@toks\fi}
\def\PY@do#1{\PY@bc{\PY@tc{\PY@ul{%
    \PY@it{\PY@bf{\PY@ff{#1}}}}}}}
\def\PY#1#2{\PY@reset\PY@toks#1+\relax+\PY@do{#2}}

\@namedef{PY@tok@w}{\def\PY@tc##1{\textcolor[rgb]{0.73,0.73,0.73}{##1}}}
\@namedef{PY@tok@c}{\let\PY@it=\textit\def\PY@tc##1{\textcolor[rgb]{0.24,0.48,0.48}{##1}}}
\@namedef{PY@tok@cp}{\def\PY@tc##1{\textcolor[rgb]{0.61,0.40,0.00}{##1}}}
\@namedef{PY@tok@k}{\let\PY@bf=\textbf\def\PY@tc##1{\textcolor[rgb]{0.00,0.50,0.00}{##1}}}
\@namedef{PY@tok@kp}{\def\PY@tc##1{\textcolor[rgb]{0.00,0.50,0.00}{##1}}}
\@namedef{PY@tok@kt}{\def\PY@tc##1{\textcolor[rgb]{0.69,0.00,0.25}{##1}}}
\@namedef{PY@tok@o}{\def\PY@tc##1{\textcolor[rgb]{0.40,0.40,0.40}{##1}}}
\@namedef{PY@tok@ow}{\let\PY@bf=\textbf\def\PY@tc##1{\textcolor[rgb]{0.67,0.13,1.00}{##1}}}
\@namedef{PY@tok@nb}{\def\PY@tc##1{\textcolor[rgb]{0.00,0.50,0.00}{##1}}}
\@namedef{PY@tok@nf}{\def\PY@tc##1{\textcolor[rgb]{0.00,0.00,1.00}{##1}}}
\@namedef{PY@tok@nc}{\let\PY@bf=\textbf\def\PY@tc##1{\textcolor[rgb]{0.00,0.00,1.00}{##1}}}
\@namedef{PY@tok@nn}{\let\PY@bf=\textbf\def\PY@tc##1{\textcolor[rgb]{0.00,0.00,1.00}{##1}}}
\@namedef{PY@tok@ne}{\let\PY@bf=\textbf\def\PY@tc##1{\textcolor[rgb]{0.80,0.25,0.22}{##1}}}
\@namedef{PY@tok@nv}{\def\PY@tc##1{\textcolor[rgb]{0.10,0.09,0.49}{##1}}}
\@namedef{PY@tok@no}{\def\PY@tc##1{\textcolor[rgb]{0.53,0.00,0.00}{##1}}}
\@namedef{PY@tok@nl}{\def\PY@tc##1{\textcolor[rgb]{0.46,0.46,0.00}{##1}}}
\@namedef{PY@tok@ni}{\let\PY@bf=\textbf\def\PY@tc##1{\textcolor[rgb]{0.44,0.44,0.44}{##1}}}
\@namedef{PY@tok@na}{\def\PY@tc##1{\textcolor[rgb]{0.41,0.47,0.13}{##1}}}
\@namedef{PY@tok@nt}{\let\PY@bf=\textbf\def\PY@tc##1{\textcolor[rgb]{0.00,0.50,0.00}{##1}}}
\@namedef{PY@tok@nd}{\def\PY@tc##1{\textcolor[rgb]{0.67,0.13,1.00}{##1}}}
\@namedef{PY@tok@s}{\def\PY@tc##1{\textcolor[rgb]{0.73,0.13,0.13}{##1}}}
\@namedef{PY@tok@sd}{\let\PY@it=\textit\def\PY@tc##1{\textcolor[rgb]{0.73,0.13,0.13}{##1}}}
\@namedef{PY@tok@si}{\let\PY@bf=\textbf\def\PY@tc##1{\textcolor[rgb]{0.64,0.35,0.47}{##1}}}
\@namedef{PY@tok@se}{\let\PY@bf=\textbf\def\PY@tc##1{\textcolor[rgb]{0.67,0.36,0.12}{##1}}}
\@namedef{PY@tok@sr}{\def\PY@tc##1{\textcolor[rgb]{0.64,0.35,0.47}{##1}}}
\@namedef{PY@tok@ss}{\def\PY@tc##1{\textcolor[rgb]{0.10,0.09,0.49}{##1}}}
\@namedef{PY@tok@sx}{\def\PY@tc##1{\textcolor[rgb]{0.00,0.50,0.00}{##1}}}
\@namedef{PY@tok@m}{\def\PY@tc##1{\textcolor[rgb]{0.40,0.40,0.40}{##1}}}
\@namedef{PY@tok@gh}{\let\PY@bf=\textbf\def\PY@tc##1{\textcolor[rgb]{0.00,0.00,0.50}{##1}}}
\@namedef{PY@tok@gu}{\let\PY@bf=\textbf\def\PY@tc##1{\textcolor[rgb]{0.50,0.00,0.50}{##1}}}
\@namedef{PY@tok@gd}{\def\PY@tc##1{\textcolor[rgb]{0.63,0.00,0.00}{##1}}}
\@namedef{PY@tok@gi}{\def\PY@tc##1{\textcolor[rgb]{0.00,0.52,0.00}{##1}}}
\@namedef{PY@tok@gr}{\def\PY@tc##1{\textcolor[rgb]{0.89,0.00,0.00}{##1}}}
\@namedef{PY@tok@ge}{\let\PY@it=\textit}
\@namedef{PY@tok@gs}{\let\PY@bf=\textbf}
\@namedef{PY@tok@ges}{\let\PY@bf=\textbf\let\PY@it=\textit}
\@namedef{PY@tok@gp}{\let\PY@bf=\textbf\def\PY@tc##1{\textcolor[rgb]{0.00,0.00,0.50}{##1}}}
\@namedef{PY@tok@go}{\def\PY@tc##1{\textcolor[rgb]{0.44,0.44,0.44}{##1}}}
\@namedef{PY@tok@gt}{\def\PY@tc##1{\textcolor[rgb]{0.00,0.27,0.87}{##1}}}
\@namedef{PY@tok@err}{\def\PY@bc##1{{\setlength{\fboxsep}{\string -\fboxrule}\fcolorbox[rgb]{1.00,0.00,0.00}{1,1,1}{\strut ##1}}}}
\@namedef{PY@tok@kc}{\let\PY@bf=\textbf\def\PY@tc##1{\textcolor[rgb]{0.00,0.50,0.00}{##1}}}
\@namedef{PY@tok@kd}{\let\PY@bf=\textbf\def\PY@tc##1{\textcolor[rgb]{0.00,0.50,0.00}{##1}}}
\@namedef{PY@tok@kn}{\let\PY@bf=\textbf\def\PY@tc##1{\textcolor[rgb]{0.00,0.50,0.00}{##1}}}
\@namedef{PY@tok@kr}{\let\PY@bf=\textbf\def\PY@tc##1{\textcolor[rgb]{0.00,0.50,0.00}{##1}}}
\@namedef{PY@tok@bp}{\def\PY@tc##1{\textcolor[rgb]{0.00,0.50,0.00}{##1}}}
\@namedef{PY@tok@fm}{\def\PY@tc##1{\textcolor[rgb]{0.00,0.00,1.00}{##1}}}
\@namedef{PY@tok@vc}{\def\PY@tc##1{\textcolor[rgb]{0.10,0.09,0.49}{##1}}}
\@namedef{PY@tok@vg}{\def\PY@tc##1{\textcolor[rgb]{0.10,0.09,0.49}{##1}}}
\@namedef{PY@tok@vi}{\def\PY@tc##1{\textcolor[rgb]{0.10,0.09,0.49}{##1}}}
\@namedef{PY@tok@vm}{\def\PY@tc##1{\textcolor[rgb]{0.10,0.09,0.49}{##1}}}
\@namedef{PY@tok@sa}{\def\PY@tc##1{\textcolor[rgb]{0.73,0.13,0.13}{##1}}}
\@namedef{PY@tok@sb}{\def\PY@tc##1{\textcolor[rgb]{0.73,0.13,0.13}{##1}}}
\@namedef{PY@tok@sc}{\def\PY@tc##1{\textcolor[rgb]{0.73,0.13,0.13}{##1}}}
\@namedef{PY@tok@dl}{\def\PY@tc##1{\textcolor[rgb]{0.73,0.13,0.13}{##1}}}
\@namedef{PY@tok@s2}{\def\PY@tc##1{\textcolor[rgb]{0.73,0.13,0.13}{##1}}}
\@namedef{PY@tok@sh}{\def\PY@tc##1{\textcolor[rgb]{0.73,0.13,0.13}{##1}}}
\@namedef{PY@tok@s1}{\def\PY@tc##1{\textcolor[rgb]{0.73,0.13,0.13}{##1}}}
\@namedef{PY@tok@mb}{\def\PY@tc##1{\textcolor[rgb]{0.40,0.40,0.40}{##1}}}
\@namedef{PY@tok@mf}{\def\PY@tc##1{\textcolor[rgb]{0.40,0.40,0.40}{##1}}}
\@namedef{PY@tok@mh}{\def\PY@tc##1{\textcolor[rgb]{0.40,0.40,0.40}{##1}}}
\@namedef{PY@tok@mi}{\def\PY@tc##1{\textcolor[rgb]{0.40,0.40,0.40}{##1}}}
\@namedef{PY@tok@il}{\def\PY@tc##1{\textcolor[rgb]{0.40,0.40,0.40}{##1}}}
\@namedef{PY@tok@mo}{\def\PY@tc##1{\textcolor[rgb]{0.40,0.40,0.40}{##1}}}
\@namedef{PY@tok@ch}{\let\PY@it=\textit\def\PY@tc##1{\textcolor[rgb]{0.24,0.48,0.48}{##1}}}
\@namedef{PY@tok@cm}{\let\PY@it=\textit\def\PY@tc##1{\textcolor[rgb]{0.24,0.48,0.48}{##1}}}
\@namedef{PY@tok@cpf}{\let\PY@it=\textit\def\PY@tc##1{\textcolor[rgb]{0.24,0.48,0.48}{##1}}}
\@namedef{PY@tok@c1}{\let\PY@it=\textit\def\PY@tc##1{\textcolor[rgb]{0.24,0.48,0.48}{##1}}}
\@namedef{PY@tok@cs}{\let\PY@it=\textit\def\PY@tc##1{\textcolor[rgb]{0.24,0.48,0.48}{##1}}}

\def\PYZbs{\char`\\}
\def\PYZus{\char`\_}
\def\PYZob{\char`\{}
\def\PYZcb{\char`\}}
\def\PYZca{\char`\^}
\def\PYZam{\char`\&}
\def\PYZlt{\char`\<}
\def\PYZgt{\char`\>}
\def\PYZsh{\char`\#}
\def\PYZpc{\char`\%}
\def\PYZdl{\char`\$}
\def\PYZhy{\char`\-}
\def\PYZsq{\char`\'}
\def\PYZdq{\char`\"}
\def\PYZti{\char`\~}
% for compatibility with earlier versions
\def\PYZat{@}
\def\PYZlb{[}
\def\PYZrb{]}
\makeatother


    % For linebreaks inside Verbatim environment from package fancyvrb.
    \makeatletter
        \newbox\Wrappedcontinuationbox
        \newbox\Wrappedvisiblespacebox
        \newcommand*\Wrappedvisiblespace {\textcolor{red}{\textvisiblespace}}
        \newcommand*\Wrappedcontinuationsymbol {\textcolor{red}{\llap{\tiny$\m@th\hookrightarrow$}}}
        \newcommand*\Wrappedcontinuationindent {3ex }
        \newcommand*\Wrappedafterbreak {\kern\Wrappedcontinuationindent\copy\Wrappedcontinuationbox}
        % Take advantage of the already applied Pygments mark-up to insert
        % potential linebreaks for TeX processing.
        %        {, <, #, %, $, ' and ": go to next line.
        %        _, }, ^, &, >, - and ~: stay at end of broken line.
        % Use of \textquotesingle for straight quote.
        \newcommand*\Wrappedbreaksatspecials {%
            \def\PYGZus{\discretionary{\char`\_}{\Wrappedafterbreak}{\char`\_}}%
            \def\PYGZob{\discretionary{}{\Wrappedafterbreak\char`\{}{\char`\{}}%
            \def\PYGZcb{\discretionary{\char`\}}{\Wrappedafterbreak}{\char`\}}}%
            \def\PYGZca{\discretionary{\char`\^}{\Wrappedafterbreak}{\char`\^}}%
            \def\PYGZam{\discretionary{\char`\&}{\Wrappedafterbreak}{\char`\&}}%
            \def\PYGZlt{\discretionary{}{\Wrappedafterbreak\char`\<}{\char`\<}}%
            \def\PYGZgt{\discretionary{\char`\>}{\Wrappedafterbreak}{\char`\>}}%
            \def\PYGZsh{\discretionary{}{\Wrappedafterbreak\char`\#}{\char`\#}}%
            \def\PYGZpc{\discretionary{}{\Wrappedafterbreak\char`\%}{\char`\%}}%
            \def\PYGZdl{\discretionary{}{\Wrappedafterbreak\char`\$}{\char`\$}}%
            \def\PYGZhy{\discretionary{\char`\-}{\Wrappedafterbreak}{\char`\-}}%
            \def\PYGZsq{\discretionary{}{\Wrappedafterbreak\textquotesingle}{\textquotesingle}}%
            \def\PYGZdq{\discretionary{}{\Wrappedafterbreak\char`\"}{\char`\"}}%
            \def\PYGZti{\discretionary{\char`\~}{\Wrappedafterbreak}{\char`\~}}%
        }
        % Some characters . , ; ? ! / are not pygmentized.
        % This macro makes them "active" and they will insert potential linebreaks
        \newcommand*\Wrappedbreaksatpunct {%
            \lccode`\~`\.\lowercase{\def~}{\discretionary{\hbox{\char`\.}}{\Wrappedafterbreak}{\hbox{\char`\.}}}%
            \lccode`\~`\,\lowercase{\def~}{\discretionary{\hbox{\char`\,}}{\Wrappedafterbreak}{\hbox{\char`\,}}}%
            \lccode`\~`\;\lowercase{\def~}{\discretionary{\hbox{\char`\;}}{\Wrappedafterbreak}{\hbox{\char`\;}}}%
            \lccode`\~`\:\lowercase{\def~}{\discretionary{\hbox{\char`\:}}{\Wrappedafterbreak}{\hbox{\char`\:}}}%
            \lccode`\~`\?\lowercase{\def~}{\discretionary{\hbox{\char`\?}}{\Wrappedafterbreak}{\hbox{\char`\?}}}%
            \lccode`\~`\!\lowercase{\def~}{\discretionary{\hbox{\char`\!}}{\Wrappedafterbreak}{\hbox{\char`\!}}}%
            \lccode`\~`\/\lowercase{\def~}{\discretionary{\hbox{\char`\/}}{\Wrappedafterbreak}{\hbox{\char`\/}}}%
            \catcode`\.\active
            \catcode`\,\active
            \catcode`\;\active
            \catcode`\:\active
            \catcode`\?\active
            \catcode`\!\active
            \catcode`\/\active
            \lccode`\~`\~
        }
    \makeatother

    \let\OriginalVerbatim=\Verbatim
    \makeatletter
    \renewcommand{\Verbatim}[1][1]{%
        %\parskip\z@skip
        \sbox\Wrappedcontinuationbox {\Wrappedcontinuationsymbol}%
        \sbox\Wrappedvisiblespacebox {\FV@SetupFont\Wrappedvisiblespace}%
        \def\FancyVerbFormatLine ##1{\hsize\linewidth
            \vtop{\raggedright\hyphenpenalty\z@\exhyphenpenalty\z@
                \doublehyphendemerits\z@\finalhyphendemerits\z@
                \strut ##1\strut}%
        }%
        % If the linebreak is at a space, the latter will be displayed as visible
        % space at end of first line, and a continuation symbol starts next line.
        % Stretch/shrink are however usually zero for typewriter font.
        \def\FV@Space {%
            \nobreak\hskip\z@ plus\fontdimen3\font minus\fontdimen4\font
            \discretionary{\copy\Wrappedvisiblespacebox}{\Wrappedafterbreak}
            {\kern\fontdimen2\font}%
        }%

        % Allow breaks at special characters using \PYG... macros.
        \Wrappedbreaksatspecials
        % Breaks at punctuation characters . , ; ? ! and / need catcode=\active
        \OriginalVerbatim[#1,codes*=\Wrappedbreaksatpunct]%
    }
    \makeatother

    % Exact colors from NB
    \definecolor{incolor}{HTML}{303F9F}
    \definecolor{outcolor}{HTML}{D84315}
    \definecolor{cellborder}{HTML}{CFCFCF}
    \definecolor{cellbackground}{HTML}{F7F7F7}

    % prompt
    \makeatletter
    \newcommand{\boxspacing}{\kern\kvtcb@left@rule\kern\kvtcb@boxsep}
    \makeatother
    \newcommand{\prompt}[4]{
        {\ttfamily\llap{{\color{#2}[#3]:\hspace{3pt}#4}}\vspace{-\baselineskip}}
    }
    

    
    % Prevent overflowing lines due to hard-to-break entities
    \sloppy
    % Setup hyperref package
    \hypersetup{
      breaklinks=true,  % so long urls are correctly broken across lines
      colorlinks=true,
      urlcolor=urlcolor,
      linkcolor=linkcolor,
      citecolor=citecolor,
      }
    % Slightly bigger margins than the latex defaults
    
    \geometry{verbose,tmargin=1in,bmargin=1in,lmargin=1in,rmargin=1in}
    
    

\begin{document}
    
    \maketitle
    
    

    
    \section{Real Option Valuation of a Bitcoin Mining
Farm}\label{real-option-valuation-of-a-bitcoin-mining-farm}

\subsubsection{Hedged Monte-Carlo (HMC) / Least-Squares Monte-Carlo
(LSMC)
Method}\label{hedged-monte-carlo-hmc-least-squares-monte-carlo-lsmc-method}

This notebook implements the real option valuation model described in
the paper \textbf{Investments in Bitcoin Mining Farms: Hedged
Monte-Carlo Method}.

The primary goal is to determine the value of the option to invest in a
Bitcoin mining farm. This approach is superior to a simple Net Present
Value (NPV) analysis because it quantifies the value of
\textbf{flexibility}---that is, the ability to delay the investment
decision until market conditions are more favorable.

The notebook is divided into four main parts: 1. \textbf{Monte Carlo
Simulation:} We simulate thousands of future paths for the key uncertain
variables: the price of Bitcoin (\(X_t\)) and the price of electricity
(\(Y_t\)). 2. \textbf{Intrinsic Value Calculation:} We calculate the
project's cash flows and its immediate exercise value (intrinsic value)
for each simulated path. 3. \textbf{Real Option Valuation:} We use a
backward induction algorithm (Least-Squares Monte Carlo) to calculate
the option's value at each point in time, which includes the value of
waiting. 4. \textbf{Analysis of Results:} We visualize the final option
value and the probability of exercising the investment option over time.

    \subsection{1. Setup and Environment
Configuration}\label{setup-and-environment-configuration}

    \begin{tcolorbox}[breakable, size=fbox, boxrule=1pt, pad at break*=1mm,colback=cellbackground, colframe=cellborder]
\prompt{In}{incolor}{1}{\boxspacing}
\begin{Verbatim}[commandchars=\\\{\}]
\PY{c+c1}{\PYZsh{} This notebook is designed to run in Google Colab.}
\PY{c+c1}{\PYZsh{} 1. Mount Google Drive to access utility files}
\PY{k+kn}{from}\PY{+w}{ }\PY{n+nn}{google}\PY{n+nn}{.}\PY{n+nn}{colab}\PY{+w}{ }\PY{k+kn}{import} \PY{n}{drive}
\PY{n}{drive}\PY{o}{.}\PY{n}{mount}\PY{p}{(}\PY{l+s+s1}{\PYZsq{}}\PY{l+s+s1}{/content/drive}\PY{l+s+s1}{\PYZsq{}}\PY{p}{)}

\PY{c+c1}{\PYZsh{} 2. Add the project folder to Python\PYZsq{}s path}
\PY{k+kn}{import}\PY{+w}{ }\PY{n+nn}{sys}
\PY{k+kn}{import}\PY{+w}{ }\PY{n+nn}{os}
\end{Verbatim}
\end{tcolorbox}

    \begin{Verbatim}[commandchars=\\\{\}]
Mounted at /content/drive
    \end{Verbatim}

    \begin{tcolorbox}[breakable, size=fbox, boxrule=1pt, pad at break*=1mm,colback=cellbackground, colframe=cellborder]
\prompt{In}{incolor}{2}{\boxspacing}
\begin{Verbatim}[commandchars=\\\{\}]
\PY{o}{!}ls\PY{+w}{ }drive/Shareddrives/AFMA\PYZhy{}HedgeMonteCarlo/HedgeMonteCarlo
\end{Verbatim}
\end{tcolorbox}

    \begin{Verbatim}[commandchars=\\\{\}]
mchedge.ipynb  MonteCarloHedge.ipynb  utils
    \end{Verbatim}

    \begin{tcolorbox}[breakable, size=fbox, boxrule=1pt, pad at break*=1mm,colback=cellbackground, colframe=cellborder]
\prompt{In}{incolor}{3}{\boxspacing}
\begin{Verbatim}[commandchars=\\\{\}]
\PY{c+c1}{\PYZsh{} \PYZhy{}\PYZhy{}\PYZhy{} IMPORTANT: CHANGE THIS PATH TO MATCH YOUR PROJECT\PYZsq{}S LOCATION \PYZhy{}\PYZhy{}\PYZhy{}}
\PY{n}{project\PYZus{}path} \PY{o}{=} \PY{l+s+s1}{\PYZsq{}}\PY{l+s+s1}{/content/drive/Shareddrives/AFMA\PYZhy{}HedgeMonteCarlo/HedgeMonteCarlo}\PY{l+s+s1}{\PYZsq{}} \PY{c+c1}{\PYZsh{} \PYZlt{}\PYZhy{}\PYZhy{}\PYZhy{} CHANGE THIS PATH}
\PY{c+c1}{\PYZsh{} \PYZhy{}\PYZhy{}\PYZhy{}\PYZhy{}\PYZhy{}\PYZhy{}\PYZhy{}\PYZhy{}\PYZhy{}\PYZhy{}\PYZhy{}\PYZhy{}\PYZhy{}\PYZhy{}\PYZhy{}\PYZhy{}\PYZhy{}\PYZhy{}\PYZhy{}\PYZhy{}\PYZhy{}\PYZhy{}\PYZhy{}\PYZhy{}\PYZhy{}\PYZhy{}\PYZhy{}\PYZhy{}\PYZhy{}\PYZhy{}\PYZhy{}\PYZhy{}\PYZhy{}\PYZhy{}\PYZhy{}\PYZhy{}\PYZhy{}\PYZhy{}\PYZhy{}\PYZhy{}\PYZhy{}\PYZhy{}\PYZhy{}\PYZhy{}\PYZhy{}\PYZhy{}\PYZhy{}\PYZhy{}\PYZhy{}\PYZhy{}\PYZhy{}\PYZhy{}\PYZhy{}\PYZhy{}\PYZhy{}\PYZhy{}\PYZhy{}\PYZhy{}\PYZhy{}\PYZhy{}\PYZhy{}\PYZhy{}\PYZhy{}\PYZhy{}\PYZhy{}\PYZhy{}\PYZhy{}}

\PY{c+c1}{\PYZsh{} Add the path to sys.path if it\PYZsq{}s not already there}
\PY{k}{if} \PY{n}{project\PYZus{}path} \PY{o+ow}{not} \PY{o+ow}{in} \PY{n}{sys}\PY{o}{.}\PY{n}{path}\PY{p}{:}
    \PY{n}{sys}\PY{o}{.}\PY{n}{path}\PY{o}{.}\PY{n}{append}\PY{p}{(}\PY{n}{project\PYZus{}path}\PY{p}{)}
\end{Verbatim}
\end{tcolorbox}

    \begin{tcolorbox}[breakable, size=fbox, boxrule=1pt, pad at break*=1mm,colback=cellbackground, colframe=cellborder]
\prompt{In}{incolor}{4}{\boxspacing}
\begin{Verbatim}[commandchars=\\\{\}]
\PY{k+kn}{import}\PY{+w}{ }\PY{n+nn}{numpy}\PY{+w}{ }\PY{k}{as}\PY{+w}{ }\PY{n+nn}{np}
\PY{k+kn}{import}\PY{+w}{ }\PY{n+nn}{pandas}\PY{+w}{ }\PY{k}{as}\PY{+w}{ }\PY{n+nn}{pd}
\PY{k+kn}{import}\PY{+w}{ }\PY{n+nn}{datetime}\PY{+w}{ }\PY{k}{as}\PY{+w}{ }\PY{n+nn}{dt}
\PY{k+kn}{import}\PY{+w}{ }\PY{n+nn}{matplotlib}\PY{n+nn}{.}\PY{n+nn}{pyplot}\PY{+w}{ }\PY{k}{as}\PY{+w}{ }\PY{n+nn}{plt}
\PY{k+kn}{import}\PY{+w}{ }\PY{n+nn}{seaborn}\PY{+w}{ }\PY{k}{as}\PY{+w}{ }\PY{n+nn}{sns}

\PY{c+c1}{\PYZsh{} Import custom utility functions for the analysis}
\PY{k+kn}{import}\PY{+w}{ }\PY{n+nn}{utils}\PY{n+nn}{.}\PY{n+nn}{random\PYZus{}functions}\PY{+w}{ }\PY{k}{as}\PY{+w}{ }\PY{n+nn}{rf}
\PY{k+kn}{import}\PY{+w}{ }\PY{n+nn}{utils}\PY{n+nn}{.}\PY{n+nn}{options}\PY{+w}{ }\PY{k}{as}\PY{+w}{ }\PY{n+nn}{op}
\PY{k+kn}{import}\PY{+w}{ }\PY{n+nn}{utils}\PY{n+nn}{.}\PY{n+nn}{optimization}\PY{+w}{ }\PY{k}{as}\PY{+w}{ }\PY{n+nn}{ot}
\end{Verbatim}
\end{tcolorbox}

    \subsection{2. Monte Carlo Simulation of Asset
Prices}\label{monte-carlo-simulation-of-asset-prices}

Here, we simulate the future price paths for Bitcoin (\(X_t\)) and
Electricity (\(Y_t\)) under the \textbf{real-world measure
\(\\mathbb{P}\)}, as discussed in Section 2.2 of the paper. We use a
multivariable geometric Brownian motion model, which accounts for the
assets' expected returns (\texttt{mu}), volatilities, and their
correlation (\texttt{cov\_matrix}). This generates
\texttt{num\_simulations} (1000) possible future scenarios for our
valuation.

    \begin{tcolorbox}[breakable, size=fbox, boxrule=1pt, pad at break*=1mm,colback=cellbackground, colframe=cellborder]
\prompt{In}{incolor}{5}{\boxspacing}
\begin{Verbatim}[commandchars=\\\{\}]
\PY{c+c1}{\PYZsh{} \PYZhy{}\PYZhy{}\PYZhy{} Simulation Parameters \PYZhy{}\PYZhy{}\PYZhy{}}
\PY{c+c1}{\PYZsh{} Note: These are placeholder values. For a real analysis, they should be derived from historical data.}
\PY{n}{num\PYZus{}assets} \PY{o}{=} \PY{l+m+mi}{2} \PY{c+c1}{\PYZsh{} Simulating two variables: Bitcoin (price\PYZus{}0) and Energy (price\PYZus{}1)}
\PY{n}{initial\PYZus{}prices} \PY{o}{=} \PY{n}{np}\PY{o}{.}\PY{n}{array}\PY{p}{(}\PY{p}{[}\PY{l+m+mi}{100}\PY{p}{,} \PY{l+m+mi}{80}\PY{p}{]}\PY{p}{)} \PY{c+c1}{\PYZsh{} Placeholder initial prices for X\PYZus{}t and Y\PYZus{}t}
\PY{n}{mu} \PY{o}{=} \PY{n}{np}\PY{o}{.}\PY{n}{array}\PY{p}{(}\PY{p}{[}\PY{l+m+mf}{0.08}\PY{p}{,} \PY{l+m+mf}{0.05}\PY{p}{]}\PY{p}{)} \PY{c+c1}{\PYZsh{} Annualized expected returns (drift)}

\PY{n}{cov\PYZus{}matrix} \PY{o}{=} \PY{n}{np}\PY{o}{.}\PY{n}{array}\PY{p}{(}\PY{p}{[}
    \PY{p}{[}\PY{l+m+mf}{0.1}\PY{p}{,} \PY{l+m+mf}{0.02}\PY{p}{]}\PY{p}{,} \PY{c+c1}{\PYZsh{} Annualized covariance matrix (volatility and correlation)}
    \PY{p}{[}\PY{l+m+mf}{0.02}\PY{p}{,} \PY{l+m+mf}{0.08}\PY{p}{]}
  \PY{p}{]}\PY{p}{)}

\PY{n}{n\PYZus{}days} \PY{o}{=} \PY{l+m+mi}{365}  \PY{c+c1}{\PYZsh{} Number of trading days in a year}
\PY{n}{num\PYZus{}simulations} \PY{o}{=} \PY{l+m+mi}{1000} \PY{c+c1}{\PYZsh{} Number of simulations (M in the paper)}

\PY{c+c1}{\PYZsh{} Convert annualized parameters to daily parameters for the simulation}
\PY{n}{daily\PYZus{}mu} \PY{o}{=} \PY{n}{mu} \PY{o}{/} \PY{n}{n\PYZus{}days}
\PY{n}{daily\PYZus{}cov} \PY{o}{=} \PY{n}{cov\PYZus{}matrix} \PY{o}{/} \PY{n}{n\PYZus{}days}

\PY{c+c1}{\PYZsh{} \PYZsh{} Number of steps for the random walk}
\PY{c+c1}{\PYZsh{} num\PYZus{}steps = n\PYZus{}days \PYZsh{} Using the projection\PYZus{}days from the preceding code}

\PY{c+c1}{\PYZsh{} \PYZsh{} Generate the multivariable random walk paths}
\PY{c+c1}{\PYZsh{} simulated\PYZus{}multivar\PYZus{}prices = rf.multivariable\PYZus{}random\PYZus{}walk(initial\PYZus{}prices, daily\PYZus{}mu, daily\PYZus{}cov, num\PYZus{}steps, num\PYZus{}simulations)}
\end{Verbatim}
\end{tcolorbox}

    \subsubsection{Define Project and Valuation
Timeframes}\label{define-project-and-valuation-timeframes}

\begin{itemize}
\tightlist
\item
  \textbf{Valuation Date:} The date of the analysis (today, \(t=0\)).
\item
  \textbf{Start/End Date:} The investment horizon \([0, T]\) during
  which the option can be exercised.
\item
  \textbf{Duration Project:} The lifetime of the mining farm once built
  (\(T'\) in the paper).''
\end{itemize}

    \begin{tcolorbox}[breakable, size=fbox, boxrule=1pt, pad at break*=1mm,colback=cellbackground, colframe=cellborder]
\prompt{In}{incolor}{6}{\boxspacing}
\begin{Verbatim}[commandchars=\\\{\}]
\PY{n}{valuation\PYZus{}date} \PY{o}{=} \PY{n}{dt}\PY{o}{.}\PY{n}{date}\PY{p}{(}\PY{l+m+mi}{2025}\PY{p}{,} \PY{l+m+mi}{1}\PY{p}{,} \PY{l+m+mi}{1}\PY{p}{)}
\PY{n}{start\PYZus{}date} \PY{o}{=} \PY{n}{dt}\PY{o}{.}\PY{n}{date}\PY{p}{(}\PY{l+m+mi}{2025}\PY{p}{,} \PY{l+m+mi}{2}\PY{p}{,} \PY{l+m+mi}{1}\PY{p}{)}
\PY{n}{end\PYZus{}date} \PY{o}{=} \PY{n}{dt}\PY{o}{.}\PY{n}{date}\PY{p}{(}\PY{l+m+mi}{2025}\PY{p}{,} \PY{l+m+mi}{5}\PY{p}{,} \PY{l+m+mi}{1}\PY{p}{)}
\PY{n}{duration\PYZus{}project} \PY{o}{=} \PY{l+m+mi}{31} \PY{c+c1}{\PYZsh{} Lifetime of the project in days (T\PYZsq{})}
\end{Verbatim}
\end{tcolorbox}

    \subsubsection{Economic Parameters for Cash Flow
Calculation}\label{economic-parameters-for-cash-flow-calculation}

These parameters correspond to the variables in the cash flow formula
from the paper: \textbf{\((X_t - \kappa Y_t)^{+} - k\)}.

\begin{itemize}
\tightlist
\item
  \texttt{rho}: Discount factor rate (\(\\rho\)).
\item
  \texttt{k}: Additional fixed operational costs (\(k\)).
\item
  \texttt{kappa}: Conversion factor for electricity needed to mine 1 BTC
  (\(\kappa\)).
\item
  \texttt{investment}: The initial sunk cost to build the farm
  (\(K\)).''
\end{itemize}

    \begin{tcolorbox}[breakable, size=fbox, boxrule=1pt, pad at break*=1mm,colback=cellbackground, colframe=cellborder]
\prompt{In}{incolor}{7}{\boxspacing}
\begin{Verbatim}[commandchars=\\\{\}]
\PY{n}{rho} \PY{o}{=} \PY{l+m+mf}{0.05} \PY{c+c1}{\PYZsh{} Discount rate}
\PY{n}{k} \PY{o}{=} \PY{l+m+mi}{10} \PY{c+c1}{\PYZsh{} Fixed costs}
\PY{n}{kappa} \PY{o}{=} \PY{l+m+mf}{0.005} \PY{c+c1}{\PYZsh{} Electricity conversion factor}
\PY{n}{investment} \PY{o}{=} \PY{l+m+mi}{3500} \PY{c+c1}{\PYZsh{} Sunk investment cost (K)}
\PY{n}{delta\PYZus{}time} \PY{o}{=} \PY{l+m+mi}{1} \PY{o}{/} \PY{l+m+mi}{365} \PY{c+c1}{\PYZsh{} Time step (Δ)}
\end{Verbatim}
\end{tcolorbox}

    \subsubsection{Generate Price Paths and Store in
DataFrames}\label{generate-price-paths-and-store-in-dataframes}

    \begin{tcolorbox}[breakable, size=fbox, boxrule=1pt, pad at break*=1mm,colback=cellbackground, colframe=cellborder]
\prompt{In}{incolor}{8}{\boxspacing}
\begin{Verbatim}[commandchars=\\\{\}]
\PY{c+c1}{\PYZsh{} The simulation must run long enough to cover the investment window plus the project\PYZsq{}s lifetime}
\PY{n}{num\PYZus{}steps} \PY{o}{=} \PY{p}{(}\PY{n}{end\PYZus{}date} \PY{o}{\PYZhy{}} \PY{n}{valuation\PYZus{}date}\PY{p}{)}\PY{o}{.}\PY{n}{days} \PY{o}{+} \PY{n}{duration\PYZus{}project}
\PY{n}{dates} \PY{o}{=} \PY{n}{np}\PY{o}{.}\PY{n}{arange}\PY{p}{(}\PY{n}{valuation\PYZus{}date}\PY{p}{,} \PY{n}{end\PYZus{}date} \PY{o}{+} \PY{n}{dt}\PY{o}{.}\PY{n}{timedelta}\PY{p}{(}\PY{n}{days}\PY{o}{=}\PY{n}{duration\PYZus{}project} \PY{o}{+} \PY{l+m+mi}{1}\PY{p}{)}\PY{p}{)}

\PY{c+c1}{\PYZsh{} Generate the multivariable random walk paths}
\PY{n}{simulated\PYZus{}multivar\PYZus{}prices} \PY{o}{=} \PY{n}{rf}\PY{o}{.}\PY{n}{multivariable\PYZus{}random\PYZus{}walk}\PY{p}{(}\PY{n}{initial\PYZus{}prices}\PY{p}{,} \PY{n}{daily\PYZus{}mu}\PY{p}{,} \PY{n}{daily\PYZus{}cov}\PY{p}{,} \PY{n}{num\PYZus{}steps}\PY{p}{,} \PY{n}{num\PYZus{}simulations}\PY{p}{)}

\PY{c+c1}{\PYZsh{} Store prices in a dictionary of DataFrames for easy access}
\PY{n}{prices} \PY{o}{=} \PY{p}{\PYZob{}}\PY{p}{\PYZcb{}}
\PY{n}{prices}\PY{p}{[}\PY{l+s+s1}{\PYZsq{}}\PY{l+s+s1}{price\PYZus{}0}\PY{l+s+s1}{\PYZsq{}}\PY{p}{]} \PY{o}{=} \PY{n}{pd}\PY{o}{.}\PY{n}{DataFrame}\PY{p}{(}\PY{n}{simulated\PYZus{}multivar\PYZus{}prices}\PY{p}{[}\PY{p}{:}\PY{p}{,} \PY{p}{:}\PY{p}{,} \PY{l+m+mi}{0}\PY{p}{]}\PY{p}{,} \PY{n}{columns}\PY{o}{=}\PY{n}{dates}\PY{p}{)} \PY{c+c1}{\PYZsh{} BTC (X\PYZus{}t)}
\PY{n}{prices}\PY{p}{[}\PY{l+s+s1}{\PYZsq{}}\PY{l+s+s1}{price\PYZus{}1}\PY{l+s+s1}{\PYZsq{}}\PY{p}{]} \PY{o}{=} \PY{n}{pd}\PY{o}{.}\PY{n}{DataFrame}\PY{p}{(}\PY{n}{simulated\PYZus{}multivar\PYZus{}prices}\PY{p}{[}\PY{p}{:}\PY{p}{,} \PY{p}{:}\PY{p}{,} \PY{l+m+mi}{1}\PY{p}{]}\PY{p}{,} \PY{n}{columns}\PY{o}{=}\PY{n}{dates}\PY{p}{)} \PY{c+c1}{\PYZsh{} Energy (Y\PYZus{}t)\PYZdq{}}
\end{Verbatim}
\end{tcolorbox}

    \subsection{3. Calculation of Project Cash Flows and Intrinsic
Value}\label{calculation-of-project-cash-flows-and-intrinsic-value}

In this section, we calculate the project's Net Present Value
(\(G(x,y)\)) for every possible investment day in each simulation, as
defined in \textbf{Equation 2.1} and approximated in \textbf{Equation
2.2} of the paper.

This value represents the total profit if the investment were made on
that specific day. It is the \textbf{"immediate exercise value"} or
\textbf{"intrinsic value"} of the real option.

    \begin{tcolorbox}[breakable, size=fbox, boxrule=1pt, pad at break*=1mm,colback=cellbackground, colframe=cellborder]
\prompt{In}{incolor}{9}{\boxspacing}
\begin{Verbatim}[commandchars=\\\{\}]
\PY{c+c1}{\PYZsh{} Create a DataFrame for the discount factor e\PYZca{}(\PYZhy{}ρt)}
\PY{n}{discount\PYZus{}factor} \PY{o}{=} \PY{n}{np}\PY{o}{.}\PY{n}{repeat}\PY{p}{(}\PY{n}{np}\PY{o}{.}\PY{n}{matrix}\PY{p}{(}\PY{p}{[}\PY{n+nb}{float}\PY{p}{(}\PY{n}{np}\PY{o}{.}\PY{n}{exp}\PY{p}{(}\PY{o}{\PYZhy{}}\PY{n}{rho} \PY{o}{*} \PY{n}{t} \PY{o}{/} \PY{l+m+mi}{365}\PY{p}{)}\PY{p}{)} \PY{k}{for} \PY{n}{t} \PY{o+ow}{in} \PY{n+nb}{range}\PY{p}{(}\PY{n}{num\PYZus{}steps} \PY{o}{+} \PY{l+m+mi}{1}\PY{p}{)}\PY{p}{]}\PY{p}{)}\PY{p}{,} \PY{n}{num\PYZus{}simulations}\PY{p}{,} \PY{n}{axis}\PY{o}{=}\PY{l+m+mi}{0}\PY{p}{)}
\PY{n}{df\PYZus{}discount\PYZus{}factor} \PY{o}{=} \PY{n}{pd}\PY{o}{.}\PY{n}{DataFrame}\PY{p}{(}\PY{n}{discount\PYZus{}factor}\PY{p}{,} \PY{n}{columns}\PY{o}{=}\PY{n}{dates}\PY{p}{)}
\end{Verbatim}
\end{tcolorbox}

    \begin{tcolorbox}[breakable, size=fbox, boxrule=1pt, pad at break*=1mm,colback=cellbackground, colframe=cellborder]
\prompt{In}{incolor}{10}{\boxspacing}
\begin{Verbatim}[commandchars=\\\{\}]
\PY{c+c1}{\PYZsh{} Calculate the daily present value of cash flows}
\PY{n}{valuation\PYZus{}dates} \PY{o}{=} \PY{n}{np}\PY{o}{.}\PY{n}{arange}\PY{p}{(}\PY{n}{start\PYZus{}date}\PY{p}{,} \PY{n}{end\PYZus{}date} \PY{o}{+} \PY{n}{dt}\PY{o}{.}\PY{n}{timedelta}\PY{p}{(}\PY{n}{days}\PY{o}{=}\PY{n}{duration\PYZus{}project} \PY{o}{+} \PY{l+m+mi}{1}\PY{p}{)}\PY{p}{)}
\PY{n}{df\PYZus{}daily\PYZus{}present\PYZus{}value} \PY{o}{=} \PY{n}{pd}\PY{o}{.}\PY{n}{DataFrame}\PY{p}{(}\PY{p}{[}\PY{p}{]}\PY{p}{,} \PY{n}{columns}\PY{o}{=}\PY{n}{valuation\PYZus{}dates}\PY{p}{)}

\PY{n}{df\PYZus{}btc\PYZus{}prices} \PY{o}{=} \PY{n}{prices}\PY{p}{[}\PY{l+s+s1}{\PYZsq{}}\PY{l+s+s1}{price\PYZus{}0}\PY{l+s+s1}{\PYZsq{}}\PY{p}{]}
\PY{n}{df\PYZus{}energy\PYZus{}prices} \PY{o}{=} \PY{n}{prices}\PY{p}{[}\PY{l+s+s1}{\PYZsq{}}\PY{l+s+s1}{price\PYZus{}1}\PY{l+s+s1}{\PYZsq{}}\PY{p}{]}

\PY{k}{for} \PY{n}{date} \PY{o+ow}{in} \PY{n}{valuation\PYZus{}dates}\PY{p}{:}
  \PY{c+c1}{\PYZsh{} This is the (X\PYZus{}t \PYZhy{} κY\PYZus{}t \PYZhy{} k) term}
  \PY{n}{df\PYZus{}daily\PYZus{}present\PYZus{}value}\PY{p}{[}\PY{n}{date}\PY{p}{]} \PY{o}{=} \PY{p}{(}\PY{n}{df\PYZus{}btc\PYZus{}prices}\PY{p}{[}\PY{n}{date}\PY{p}{]} \PY{o}{\PYZhy{}} \PY{n}{kappa} \PY{o}{*} \PY{n}{df\PYZus{}energy\PYZus{}prices}\PY{p}{[}\PY{n}{date}\PY{p}{]} \PY{o}{\PYZhy{}} \PY{n}{k}\PY{p}{)}

  \PY{c+c1}{\PYZsh{} Apply the discount factor e\PYZca{}(\PYZhy{}ρ(t\PYZhy{}τ))}
  \PY{n}{df\PYZus{}daily\PYZus{}present\PYZus{}value}\PY{p}{[}\PY{n}{date}\PY{p}{]} \PY{o}{*}\PY{o}{=} \PY{n}{df\PYZus{}discount\PYZus{}factor}\PY{p}{[}\PY{n}{date}\PY{p}{]}

  \PY{c+c1}{\PYZsh{} The firm can switch to an idle state, so negative cash flows are zeroed out: (X\PYZus{}t \PYZhy{} κY\PYZus{}t)\PYZca{}+}
  \PY{n}{df\PYZus{}daily\PYZus{}present\PYZus{}value}\PY{o}{.}\PY{n}{loc}\PY{p}{[}\PY{n}{df\PYZus{}daily\PYZus{}present\PYZus{}value}\PY{p}{[}\PY{n}{date}\PY{p}{]} \PY{o}{\PYZlt{}} \PY{l+m+mi}{0}\PY{p}{,} \PY{n}{date}\PY{p}{]} \PY{o}{=} \PY{l+m+mi}{0}
\end{Verbatim}
\end{tcolorbox}

    \begin{tcolorbox}[breakable, size=fbox, boxrule=1pt, pad at break*=1mm,colback=cellbackground, colframe=cellborder]
\prompt{In}{incolor}{11}{\boxspacing}
\begin{Verbatim}[commandchars=\\\{\}]
\PY{c+c1}{\PYZsh{} Calculate the total Net Present Value (G) by summing cash flows over the project\PYZsq{}s life (T\PYZsq{})}
\PY{c+c1}{\PYZsh{} This is done efficiently using a forward\PYZhy{}looking rolling sum.}

\PY{c+c1}{\PYZsh{} \PYZhy{}\PYZhy{}\PYZhy{} 1. Define the full date range needed for the calculation \PYZhy{}\PYZhy{}\PYZhy{}}
\PY{n}{extended\PYZus{}end\PYZus{}date} \PY{o}{=} \PY{n}{end\PYZus{}date} \PY{o}{+} \PY{n}{dt}\PY{o}{.}\PY{n}{timedelta}\PY{p}{(}\PY{n}{days}\PY{o}{=}\PY{n}{duration\PYZus{}project}\PY{p}{)}

\PY{c+c1}{\PYZsh{} \PYZhy{}\PYZhy{}\PYZhy{} 2. Slice the required data ONCE \PYZhy{}\PYZhy{}\PYZhy{}}
\PY{n}{relevant\PYZus{}data} \PY{o}{=} \PY{n}{df\PYZus{}daily\PYZus{}present\PYZus{}value}\PY{o}{.}\PY{n}{loc}\PY{p}{[}\PY{p}{:}\PY{p}{,} \PY{n}{start\PYZus{}date}\PY{p}{:}\PY{n}{extended\PYZus{}end\PYZus{}date}\PY{p}{]}

\PY{c+c1}{\PYZsh{} \PYZhy{}\PYZhy{}\PYZhy{} 3. Calculate the forward\PYZhy{}looking rolling sum in a single operation \PYZhy{}\PYZhy{}\PYZhy{}}
\PY{n}{df\PYZus{}present\PYZus{}value} \PY{o}{=} \PY{n}{relevant\PYZus{}data}\PY{o}{.}\PY{n}{T}\PY{o}{.}\PY{n}{rolling}\PY{p}{(}\PY{n}{window}\PY{o}{=}\PY{n}{duration\PYZus{}project}\PY{p}{)}\PY{o}{.}\PY{n}{sum}\PY{p}{(}\PY{p}{)}\PY{o}{.}\PY{n}{shift}\PY{p}{(}\PY{o}{\PYZhy{}}\PY{n}{duration\PYZus{}project}\PY{p}{)}\PY{o}{.}\PY{n}{dropna}\PY{p}{(}\PY{p}{)}\PY{o}{.}\PY{n}{T}
\end{Verbatim}
\end{tcolorbox}

    \subsubsection{Visualize the Simulated NPV
Paths}\label{visualize-the-simulated-npv-paths}

The plot below shows the mean of all 1000 simulated NPV paths (in blue)
and the first 5 individual paths (in orange). This illustrates the high
degree of uncertainty in the project's future value.

    \begin{tcolorbox}[breakable, size=fbox, boxrule=1pt, pad at break*=1mm,colback=cellbackground, colframe=cellborder]
\prompt{In}{incolor}{12}{\boxspacing}
\begin{Verbatim}[commandchars=\\\{\}]
\PY{n}{plt}\PY{o}{.}\PY{n}{figure}\PY{p}{(}\PY{n}{figsize}\PY{o}{=}\PY{p}{(}\PY{l+m+mi}{12}\PY{p}{,} \PY{l+m+mi}{7}\PY{p}{)}\PY{p}{)}
\PY{n}{df\PYZus{}present\PYZus{}value}\PY{o}{.}\PY{n}{mean}\PY{p}{(}\PY{p}{)}\PY{o}{.}\PY{n}{plot}\PY{p}{(}\PY{n}{legend}\PY{o}{=}\PY{k+kc}{False}\PY{p}{,} \PY{n}{title}\PY{o}{=}\PY{l+s+s1}{\PYZsq{}}\PY{l+s+s1}{Simulated Net Present Value (G) Paths}\PY{l+s+s1}{\PYZsq{}}\PY{p}{)}
\PY{n}{df\PYZus{}present\PYZus{}value}\PY{o}{.}\PY{n}{head}\PY{p}{(}\PY{p}{)}\PY{o}{.}\PY{n}{T}\PY{o}{.}\PY{n}{plot}\PY{p}{(}\PY{n}{legend}\PY{o}{=}\PY{k+kc}{False}\PY{p}{)}
\PY{n}{plt}\PY{o}{.}\PY{n}{xlabel}\PY{p}{(}\PY{l+s+s1}{\PYZsq{}}\PY{l+s+s1}{Date}\PY{l+s+s1}{\PYZsq{}}\PY{p}{)}
\PY{n}{plt}\PY{o}{.}\PY{n}{ylabel}\PY{p}{(}\PY{l+s+s1}{\PYZsq{}}\PY{l+s+s1}{NPV (G)}\PY{l+s+s1}{\PYZsq{}}\PY{p}{)}
\PY{n}{plt}\PY{o}{.}\PY{n}{grid}\PY{p}{(}\PY{k+kc}{True}\PY{p}{)}
\PY{n}{plt}\PY{o}{.}\PY{n}{show}\PY{p}{(}\PY{p}{)}
\end{Verbatim}
\end{tcolorbox}

    \begin{center}
    \adjustimage{max size={0.9\linewidth}{0.9\paperheight}}{MonteCarloHedge_files/MonteCarloHedge_19_0.png}
    \end{center}
    { \hspace*{\fill} \\}
    
    \begin{center}
    \adjustimage{max size={0.9\linewidth}{0.9\paperheight}}{MonteCarloHedge_files/MonteCarloHedge_19_1.png}
    \end{center}
    { \hspace*{\fill} \\}
    
    \subsection{4. Real Option Valuation via Backward Induction
(LSMC)}\label{real-option-valuation-via-backward-induction-lsmc}

This is the core of the algorithm. We solve the optimal stopping problem
from \textbf{Equation 2.3} by working backward in time from the option's
expiry date.

At each time step \(t_j\), we compare: 1. \textbf{Intrinsic Value:} The
value of exercising immediately, \(G(X_{t_j}, Y_{t_j})\). 2.
\textbf{Continuation Value:} The expected value of holding the option,
estimated by regressing the option's value at \(t_{j+1}\) against the
asset prices at \(t_j\). This corresponds to \(CV(j, x, y)\) in
\textbf{Equation 2.5}.

The option's value is then the maximum of these two, as shown in
\textbf{Equation 2.8}: \(V(j, m) = \max(CV, G)\).

    \begin{tcolorbox}[breakable, size=fbox, boxrule=1pt, pad at break*=1mm,colback=cellbackground, colframe=cellborder]
\prompt{In}{incolor}{13}{\boxspacing}
\begin{Verbatim}[commandchars=\\\{\}]
\PY{c+c1}{\PYZsh{} Prepare DataFrames to store the results of the backward induction}
\PY{n}{dates\PYZus{}project\PYZus{}window} \PY{o}{=} \PY{n}{np}\PY{o}{.}\PY{n}{arange}\PY{p}{(}\PY{n}{start\PYZus{}date}\PY{p}{,} \PY{n}{end\PYZus{}date} \PY{o}{+} \PY{n}{dt}\PY{o}{.}\PY{n}{timedelta}\PY{p}{(}\PY{n}{days}\PY{o}{=}\PY{l+m+mi}{1}\PY{p}{)}\PY{p}{)}
\PY{n}{df\PYZus{}project\PYZus{}value} \PY{o}{=} \PY{n}{pd}\PY{o}{.}\PY{n}{DataFrame}\PY{p}{(}\PY{p}{[}\PY{p}{]}\PY{p}{,} \PY{n}{columns}\PY{o}{=}\PY{n}{dates\PYZus{}project\PYZus{}window}\PY{p}{)}
\PY{n}{df\PYZus{}option\PYZus{}value} \PY{o}{=} \PY{n}{pd}\PY{o}{.}\PY{n}{DataFrame}\PY{p}{(}\PY{p}{[}\PY{p}{]}\PY{p}{,} \PY{n}{columns}\PY{o}{=}\PY{n}{dates\PYZus{}project\PYZus{}window}\PY{p}{)}
\PY{n}{df\PYZus{}call\PYZus{}value} \PY{o}{=} \PY{n}{pd}\PY{o}{.}\PY{n}{DataFrame}\PY{p}{(}\PY{p}{[}\PY{p}{]}\PY{p}{,} \PY{n}{columns}\PY{o}{=}\PY{n}{dates\PYZus{}project\PYZus{}window}\PY{p}{)} \PY{c+c1}{\PYZsh{} This will store the continuation value}
\PY{n}{df\PYZus{}intrinsic\PYZus{}value} \PY{o}{=} \PY{n}{pd}\PY{o}{.}\PY{n}{DataFrame}\PY{p}{(}\PY{p}{[}\PY{p}{]}\PY{p}{,} \PY{n}{columns}\PY{o}{=}\PY{n}{dates\PYZus{}project\PYZus{}window}\PY{p}{)}
\end{Verbatim}
\end{tcolorbox}

    \begin{tcolorbox}[breakable, size=fbox, boxrule=1pt, pad at break*=1mm,colback=cellbackground, colframe=cellborder]
\prompt{In}{incolor}{14}{\boxspacing}
\begin{Verbatim}[commandchars=\\\{\}]
\PY{c+c1}{\PYZsh{} Helper function to get the asset prices for a specific date}

\PY{k}{def}\PY{+w}{ }\PY{n+nf}{obtain\PYZus{}prices}\PY{p}{(}\PY{n}{prices}\PY{p}{,} \PY{n}{date}\PY{p}{)}\PY{p}{:}
    \PY{n}{df\PYZus{}prices\PYZus{}aux} \PY{o}{=} \PY{n}{pd}\PY{o}{.}\PY{n}{DataFrame}\PY{p}{(}\PY{p}{[}\PY{p}{]}\PY{p}{)}
    \PY{k}{for} \PY{n}{price} \PY{o+ow}{in} \PY{n}{prices}\PY{p}{:}
        \PY{n}{df\PYZus{}prices\PYZus{}aux} \PY{o}{=} \PY{n}{pd}\PY{o}{.}\PY{n}{concat}\PY{p}{(}\PY{p}{[}\PY{n}{df\PYZus{}prices\PYZus{}aux}\PY{p}{,} \PY{n}{prices}\PY{p}{[}\PY{n}{price}\PY{p}{]}\PY{p}{[}\PY{p}{[}\PY{n}{date}\PY{p}{]}\PY{p}{]}\PY{p}{]}\PY{p}{,} \PY{n}{axis}\PY{o}{=}\PY{l+m+mi}{1}\PY{p}{)}
    \PY{n}{df\PYZus{}prices\PYZus{}aux}\PY{o}{.}\PY{n}{columns} \PY{o}{=} \PY{n+nb}{list}\PY{p}{(}\PY{n}{prices}\PY{o}{.}\PY{n}{keys}\PY{p}{(}\PY{p}{)}\PY{p}{)}
    \PY{k}{return} \PY{n}{df\PYZus{}prices\PYZus{}aux}
\end{Verbatim}
\end{tcolorbox}

    \begin{tcolorbox}[breakable, size=fbox, boxrule=1pt, pad at break*=1mm,colback=cellbackground, colframe=cellborder]
\prompt{In}{incolor}{15}{\boxspacing}
\begin{Verbatim}[commandchars=\\\{\}]
\PY{c+c1}{\PYZsh{} \PYZhy{}\PYZhy{}\PYZhy{} Backward Induction Loop \PYZhy{}\PYZhy{}\PYZhy{}}
\PY{n}{date\PYZus{}nxt} \PY{o}{=} \PY{n}{dates\PYZus{}project\PYZus{}window}\PY{p}{[}\PY{o}{\PYZhy{}}\PY{l+m+mi}{1}\PY{p}{]}
\PY{k}{for} \PY{n}{date} \PY{o+ow}{in} \PY{n+nb}{reversed}\PY{p}{(}\PY{n}{dates\PYZus{}project\PYZus{}window}\PY{p}{)}\PY{p}{:}
    \PY{c+c1}{\PYZsh{} 1. Calculate the project value (NPV \PYZhy{} Investment) and intrinsic value for the current date}
    \PY{n}{df\PYZus{}project\PYZus{}value}\PY{p}{[}\PY{n}{date}\PY{p}{]} \PY{o}{=} \PY{n}{df\PYZus{}present\PYZus{}value}\PY{p}{[}\PY{n}{date}\PY{p}{]} \PY{o}{\PYZhy{}} \PY{n}{investment}
    \PY{n}{df\PYZus{}intrinsic\PYZus{}value}\PY{p}{[}\PY{n}{date}\PY{p}{]} \PY{o}{=} \PY{n}{df\PYZus{}project\PYZus{}value}\PY{p}{[}\PY{n}{date}\PY{p}{]}\PY{o}{.}\PY{n}{copy}\PY{p}{(}\PY{p}{)}
    \PY{n}{df\PYZus{}intrinsic\PYZus{}value}\PY{o}{.}\PY{n}{loc}\PY{p}{[}\PY{n}{df\PYZus{}intrinsic\PYZus{}value}\PY{p}{[}\PY{n}{date}\PY{p}{]} \PY{o}{\PYZlt{}} \PY{l+m+mi}{0}\PY{p}{,} \PY{n}{date}\PY{p}{]} \PY{o}{=} \PY{l+m+mi}{0} \PY{c+c1}{\PYZsh{} Intrinsic value cannot be negative}

    \PY{c+c1}{\PYZsh{} 2. Set the terminal condition at the last date (T)}
    \PY{k}{if} \PY{n}{date} \PY{o}{==} \PY{n}{end\PYZus{}date}\PY{p}{:}
        \PY{n}{df\PYZus{}option\PYZus{}value}\PY{p}{[}\PY{n}{date}\PY{p}{]} \PY{o}{=} \PY{n}{df\PYZus{}intrinsic\PYZus{}value}\PY{p}{[}\PY{n}{date}\PY{p}{]}\PY{o}{.}\PY{n}{copy}\PY{p}{(}\PY{p}{)}
        \PY{n}{df\PYZus{}call\PYZus{}value}\PY{p}{[}\PY{n}{date}\PY{p}{]} \PY{o}{=} \PY{n}{df\PYZus{}option\PYZus{}value}\PY{p}{[}\PY{n}{date}\PY{p}{]}
        \PY{n}{df\PYZus{}prices\PYZus{}nxt} \PY{o}{=} \PY{n}{obtain\PYZus{}prices}\PY{p}{(}\PY{n}{prices}\PY{p}{,} \PY{n}{date}\PY{p}{)}

    \PY{c+c1}{\PYZsh{} 3. For all other dates, calculate the continuation value}
    \PY{k}{else}\PY{p}{:}
        \PY{n}{df\PYZus{}prices} \PY{o}{=} \PY{n}{obtain\PYZus{}prices}\PY{p}{(}\PY{n}{prices}\PY{p}{,} \PY{n}{date}\PY{p}{)}
        \PY{n}{df\PYZus{}return} \PY{o}{=} \PY{n}{np}\PY{o}{.}\PY{n}{log}\PY{p}{(}\PY{n}{df\PYZus{}prices\PYZus{}nxt} \PY{o}{/} \PY{n}{df\PYZus{}prices}\PY{p}{)}

        \PY{c+c1}{\PYZsh{} Create the basis functions (C\PYZus{}a(x,y) in the paper) for the regression}
        \PY{n}{df\PYZus{}k} \PY{o}{=} \PY{n}{op}\PY{o}{.}\PY{n}{base\PYZus{}bs}\PY{p}{(}\PY{n}{df\PYZus{}prices}\PY{p}{,} \PY{n}{df\PYZus{}return}\PY{p}{,} \PY{n}{delta\PYZus{}time}\PY{p}{,} \PY{n}{rho}\PY{p}{)}\PY{o}{.}\PY{n}{fillna}\PY{p}{(}\PY{l+m+mi}{0}\PY{p}{)}
        \PY{n}{df\PYZus{}h} \PY{o}{=} \PY{n}{op}\PY{o}{.}\PY{n}{base\PYZus{}bs\PYZus{}delta}\PY{p}{(}\PY{n}{df\PYZus{}prices}\PY{p}{,} \PY{n}{df\PYZus{}return}\PY{p}{,} \PY{n}{delta\PYZus{}time}\PY{p}{,} \PY{n}{rho}\PY{p}{)}\PY{o}{.}\PY{n}{fillna}\PY{p}{(}\PY{l+m+mi}{0}\PY{p}{)}
        \PY{n}{df\PYZus{}m} \PY{o}{=} \PY{n}{pd}\PY{o}{.}\PY{n}{concat}\PY{p}{(}\PY{p}{(}\PY{n}{df\PYZus{}k}\PY{p}{,} \PY{n}{df\PYZus{}h}\PY{p}{)}\PY{p}{,} \PY{n}{axis}\PY{o}{=}\PY{l+m+mi}{1}\PY{p}{)}

        \PY{c+c1}{\PYZsh{} Known option values from the next time step}
        \PY{n}{df\PYZus{}option\PYZus{}nxt} \PY{o}{=} \PY{n}{df\PYZus{}option\PYZus{}value}\PY{p}{[}\PY{p}{[}\PY{n}{date\PYZus{}nxt}\PY{p}{]}\PY{p}{]}\PY{o}{.}\PY{n}{copy}\PY{p}{(}\PY{p}{)}
        \PY{n}{df\PYZus{}option\PYZus{}nxt}\PY{o}{.}\PY{n}{rename}\PY{p}{(}\PY{n}{columns}\PY{o}{=}\PY{p}{\PYZob{}}\PY{n}{date\PYZus{}nxt}\PY{p}{:} \PY{l+s+s1}{\PYZsq{}}\PY{l+s+s1}{option}\PY{l+s+s1}{\PYZsq{}}\PY{p}{\PYZcb{}}\PY{p}{,} \PY{n}{inplace}\PY{o}{=}\PY{k+kc}{True}\PY{p}{)}

        \PY{c+c1}{\PYZsh{} Run regression to find coefficients (γ\PYZus{}a\PYZca{}j) and estimate continuation value}
        \PY{n}{mt\PYZus{}delta} \PY{o}{=} \PY{n}{ot}\PY{o}{.}\PY{n}{optimization}\PY{p}{(}\PY{n}{df\PYZus{}m}\PY{p}{,} \PY{n}{df\PYZus{}option\PYZus{}nxt}\PY{p}{)}
        \PY{n}{option\PYZus{}aux} \PY{o}{=} \PY{n}{np}\PY{o}{.}\PY{n}{matrix}\PY{p}{(}\PY{n}{df\PYZus{}k}\PY{p}{)} \PY{o}{@} \PY{n}{mt\PYZus{}delta}\PY{p}{[}\PY{p}{:} \PY{n}{df\PYZus{}k}\PY{o}{.}\PY{n}{shape}\PY{p}{[}\PY{l+m+mi}{1}\PY{p}{]}\PY{p}{,} \PY{p}{:}\PY{p}{]}
        \PY{n}{df\PYZus{}call\PYZus{}value}\PY{p}{[}\PY{n}{date}\PY{p}{]} \PY{o}{=} \PY{n}{option\PYZus{}aux} \PY{c+c1}{\PYZsh{} Store the continuation value}

        \PY{c+c1}{\PYZsh{} 4. Determine the option value: V = max(Continuation, Intrinsic)}
        \PY{n}{df\PYZus{}option\PYZus{}value}\PY{p}{[}\PY{n}{date}\PY{p}{]} \PY{o}{=} \PY{n}{option\PYZus{}aux}
        \PY{n}{df\PYZus{}option\PYZus{}value}\PY{o}{.}\PY{n}{loc}\PY{p}{[}\PY{n}{df\PYZus{}option\PYZus{}value}\PY{p}{[}\PY{n}{date}\PY{p}{]} \PY{o}{\PYZlt{}} \PY{n}{df\PYZus{}intrinsic\PYZus{}value}\PY{p}{[}\PY{n}{date}\PY{p}{]}\PY{p}{,} \PY{n}{date}\PY{p}{]} \PY{o}{=} \PY{n}{df\PYZus{}intrinsic\PYZus{}value}\PY{p}{[}\PY{n}{date}\PY{p}{]}

        \PY{c+c1}{\PYZsh{} Update for the next iteration}
        \PY{n}{date\PYZus{}nxt} \PY{o}{=} \PY{n}{date}
        \PY{n}{df\PYZus{}prices\PYZus{}nxt} \PY{o}{=} \PY{n}{df\PYZus{}prices}\PY{o}{.}\PY{n}{copy}\PY{p}{(}\PY{p}{)}
\end{Verbatim}
\end{tcolorbox}

    \subsection{5. Analysis of Results}\label{analysis-of-results}

Now we visualize the outcomes of the valuation.

\subsubsection{Probability of Early
Exercise}\label{probability-of-early-exercise}

This plot shows the probability, at each point in time, that it is
optimal to exercise the investment option immediately. This occurs in
simulations where the intrinsic value is greater than the continuation
value.

    \begin{tcolorbox}[breakable, size=fbox, boxrule=1pt, pad at break*=1mm,colback=cellbackground, colframe=cellborder]
\prompt{In}{incolor}{16}{\boxspacing}
\begin{Verbatim}[commandchars=\\\{\}]
\PY{c+c1}{\PYZsh{} Calculate the probability of exercise by checking where Continuation Value \PYZlt{} Intrinsic Value}
\PY{n}{df\PYZus{}prob\PYZus{}aux} \PY{o}{=} \PY{n}{df\PYZus{}option\PYZus{}value}\PY{o}{.}\PY{n}{copy}\PY{p}{(}\PY{p}{)} \PY{o}{*} \PY{l+m+mi}{0}
\PY{n}{df\PYZus{}prob\PYZus{}aux}\PY{p}{[}\PY{n}{df\PYZus{}call\PYZus{}value} \PY{o}{\PYZlt{}} \PY{n}{df\PYZus{}intrinsic\PYZus{}value}\PY{p}{]} \PY{o}{=} \PY{l+m+mi}{1}

\PY{n}{plt}\PY{o}{.}\PY{n}{figure}\PY{p}{(}\PY{n}{figsize}\PY{o}{=}\PY{p}{(}\PY{l+m+mi}{12}\PY{p}{,} \PY{l+m+mi}{7}\PY{p}{)}\PY{p}{)}
\PY{p}{(}\PY{n}{df\PYZus{}prob\PYZus{}aux}\PY{o}{.}\PY{n}{sum}\PY{p}{(}\PY{p}{)} \PY{o}{/} \PY{n+nb}{len}\PY{p}{(}\PY{n}{df\PYZus{}prob\PYZus{}aux}\PY{p}{)}\PY{p}{)}\PY{o}{.}\PY{n}{plot}\PY{p}{(}\PY{n}{title}\PY{o}{=}\PY{l+s+s1}{\PYZsq{}}\PY{l+s+s1}{Probability of Early Exercise}\PY{l+s+s1}{\PYZsq{}}\PY{p}{)}
\PY{n}{plt}\PY{o}{.}\PY{n}{xlabel}\PY{p}{(}\PY{l+s+s1}{\PYZsq{}}\PY{l+s+s1}{Date}\PY{l+s+s1}{\PYZsq{}}\PY{p}{)}
\PY{n}{plt}\PY{o}{.}\PY{n}{ylabel}\PY{p}{(}\PY{l+s+s1}{\PYZsq{}}\PY{l+s+s1}{Probability}\PY{l+s+s1}{\PYZsq{}}\PY{p}{)}
\PY{n}{plt}\PY{o}{.}\PY{n}{grid}\PY{p}{(}\PY{k+kc}{True}\PY{p}{)}
\PY{n}{plt}\PY{o}{.}\PY{n}{show}\PY{p}{(}\PY{p}{)}
\end{Verbatim}
\end{tcolorbox}

    \begin{center}
    \adjustimage{max size={0.9\linewidth}{0.9\paperheight}}{MonteCarloHedge_files/MonteCarloHedge_25_0.png}
    \end{center}
    { \hspace*{\fill} \\}
    
    \begin{tcolorbox}[breakable, size=fbox, boxrule=1pt, pad at break*=1mm,colback=cellbackground, colframe=cellborder]
\prompt{In}{incolor}{17}{\boxspacing}
\begin{Verbatim}[commandchars=\\\{\}]
\PY{n}{df\PYZus{}intrinsic\PYZus{}value}\PY{o}{.}\PY{n}{mean}\PY{p}{(}\PY{p}{)}\PY{o}{.}\PY{n}{plot}\PY{p}{(}\PY{p}{)}
\end{Verbatim}
\end{tcolorbox}

            \begin{tcolorbox}[breakable, size=fbox, boxrule=.5pt, pad at break*=1mm, opacityfill=0]
\prompt{Out}{outcolor}{17}{\boxspacing}
\begin{Verbatim}[commandchars=\\\{\}]
<Axes: >
\end{Verbatim}
\end{tcolorbox}
        
    \begin{center}
    \adjustimage{max size={0.9\linewidth}{0.9\paperheight}}{MonteCarloHedge_files/MonteCarloHedge_26_1.png}
    \end{center}
    { \hspace*{\fill} \\}
    
    \begin{tcolorbox}[breakable, size=fbox, boxrule=1pt, pad at break*=1mm,colback=cellbackground, colframe=cellborder]
\prompt{In}{incolor}{18}{\boxspacing}
\begin{Verbatim}[commandchars=\\\{\}]
\PY{n}{df\PYZus{}option\PYZus{}value}\PY{o}{.}\PY{n}{mean}\PY{p}{(}\PY{p}{)}\PY{o}{.}\PY{n}{plot}\PY{p}{(}\PY{p}{)}
\end{Verbatim}
\end{tcolorbox}

            \begin{tcolorbox}[breakable, size=fbox, boxrule=.5pt, pad at break*=1mm, opacityfill=0]
\prompt{Out}{outcolor}{18}{\boxspacing}
\begin{Verbatim}[commandchars=\\\{\}]
<Axes: >
\end{Verbatim}
\end{tcolorbox}
        
    \begin{center}
    \adjustimage{max size={0.9\linewidth}{0.9\paperheight}}{MonteCarloHedge_files/MonteCarloHedge_27_1.png}
    \end{center}
    { \hspace*{\fill} \\}
    
    \subsubsection{Option Value vs.~Intrinsic
Value}\label{option-value-vs.-intrinsic-value}

This is the key result. The plot compares: - \textbf{Option Value
(Blue):} The total value of the investment opportunity, including the
value of flexibility to wait. - \textbf{Intrinsic Value (Orange):} The
value of investing immediately.

The difference between the two lines represents the \textbf{Time Value}
or \textbf{Flexibility Value} of the option. As you can see, the total
option value is always greater than or equal to the intrinsic value,
demonstrating the worth of being able to delay the investment.

    \begin{tcolorbox}[breakable, size=fbox, boxrule=1pt, pad at break*=1mm,colback=cellbackground, colframe=cellborder]
\prompt{In}{incolor}{19}{\boxspacing}
\begin{Verbatim}[commandchars=\\\{\}]
\PY{n}{plt}\PY{o}{.}\PY{n}{figure}\PY{p}{(}\PY{n}{figsize}\PY{o}{=}\PY{p}{(}\PY{l+m+mi}{12}\PY{p}{,} \PY{l+m+mi}{7}\PY{p}{)}\PY{p}{)}
\PY{n}{df\PYZus{}option\PYZus{}value}\PY{o}{.}\PY{n}{mean}\PY{p}{(}\PY{p}{)}\PY{o}{.}\PY{n}{plot}\PY{p}{(}\PY{n}{label}\PY{o}{=}\PY{l+s+s1}{\PYZsq{}}\PY{l+s+s1}{Option Value (Total Value)}\PY{l+s+s1}{\PYZsq{}}\PY{p}{)}
\PY{n}{df\PYZus{}intrinsic\PYZus{}value}\PY{o}{.}\PY{n}{mean}\PY{p}{(}\PY{p}{)}\PY{o}{.}\PY{n}{plot}\PY{p}{(}\PY{n}{label}\PY{o}{=}\PY{l+s+s1}{\PYZsq{}}\PY{l+s+s1}{Intrinsic Value (Immediate Exercise)}\PY{l+s+s1}{\PYZsq{}}\PY{p}{)}
\PY{n}{plt}\PY{o}{.}\PY{n}{title}\PY{p}{(}\PY{l+s+s1}{\PYZsq{}}\PY{l+s+s1}{Real Option Value vs. Intrinsic Value}\PY{l+s+s1}{\PYZsq{}}\PY{p}{)}
\PY{n}{plt}\PY{o}{.}\PY{n}{xlabel}\PY{p}{(}\PY{l+s+s1}{\PYZsq{}}\PY{l+s+s1}{Date}\PY{l+s+s1}{\PYZsq{}}\PY{p}{)}
\PY{n}{plt}\PY{o}{.}\PY{n}{ylabel}\PY{p}{(}\PY{l+s+s1}{\PYZsq{}}\PY{l+s+s1}{Value}\PY{l+s+s1}{\PYZsq{}}\PY{p}{)}
\PY{n}{plt}\PY{o}{.}\PY{n}{legend}\PY{p}{(}\PY{p}{)}
\PY{n}{plt}\PY{o}{.}\PY{n}{grid}\PY{p}{(}\PY{k+kc}{True}\PY{p}{)}
\PY{n}{plt}\PY{o}{.}\PY{n}{show}\PY{p}{(}\PY{p}{)}
\end{Verbatim}
\end{tcolorbox}

    \begin{center}
    \adjustimage{max size={0.9\linewidth}{0.9\paperheight}}{MonteCarloHedge_files/MonteCarloHedge_29_0.png}
    \end{center}
    { \hspace*{\fill} \\}
    
    \begin{tcolorbox}[breakable, size=fbox, boxrule=1pt, pad at break*=1mm,colback=cellbackground, colframe=cellborder]
\prompt{In}{incolor}{22}{\boxspacing}
\begin{Verbatim}[commandchars=\\\{\}]
\PY{c+c1}{\PYZsh{} The final value of the option at the beginning of the investment window}
\PY{n}{time\PYZus{}value} \PY{o}{=} \PY{p}{(}\PY{n}{df\PYZus{}option\PYZus{}value}\PY{p}{[}\PY{n}{pd}\PY{o}{.}\PY{n}{to\PYZus{}datetime}\PY{p}{(}\PY{n}{start\PYZus{}date}\PY{p}{)}\PY{p}{]} \PY{o}{\PYZhy{}} \PY{n}{df\PYZus{}intrinsic\PYZus{}value}\PY{p}{[}\PY{n}{pd}\PY{o}{.}\PY{n}{to\PYZus{}datetime}\PY{p}{(}\PY{n}{start\PYZus{}date}\PY{p}{)}\PY{p}{]}\PY{p}{)}\PY{o}{.}\PY{n}{mean}\PY{p}{(}\PY{p}{)}
\PY{n+nb}{print}\PY{p}{(}\PY{l+s+sa}{f}\PY{l+s+s2}{\PYZdq{}}\PY{l+s+s2}{The average Time Value (Flexibility Value) of the option at the start is: }\PY{l+s+si}{\PYZob{}}\PY{n}{time\PYZus{}value}\PY{l+s+si}{:}\PY{l+s+s2}{.2f}\PY{l+s+si}{\PYZcb{}}\PY{l+s+s2}{\PYZdq{}}\PY{p}{)}
\end{Verbatim}
\end{tcolorbox}

    \begin{Verbatim}[commandchars=\\\{\}]
The average Time Value (Flexibility Value) of the option at the start is: 167.49
    \end{Verbatim}

    \begin{tcolorbox}[breakable, size=fbox, boxrule=1pt, pad at break*=1mm,colback=cellbackground, colframe=cellborder]
\prompt{In}{incolor}{24}{\boxspacing}
\begin{Verbatim}[commandchars=\\\{\}]
\PY{o}{!}jupyter\PY{+w}{ }nbconvert\PY{+w}{ }\PYZhy{}\PYZhy{}to\PY{+w}{ }latex\PY{+w}{ }MonteCarloHedge.ipynb
\end{Verbatim}
\end{tcolorbox}

    \begin{Verbatim}[commandchars=\\\{\}]
[NbConvertApp] WARNING | pattern 'MonteCarloHedge.ipynb' matched no files
This application is used to convert notebook files (*.ipynb)
        to various other formats.

        WARNING: THE COMMANDLINE INTERFACE MAY CHANGE IN FUTURE RELEASES.

Options
=======
The options below are convenience aliases to configurable class-options,
as listed in the "Equivalent to" description-line of the aliases.
To see all configurable class-options for some <cmd>, use:
    <cmd> --help-all

--debug
    set log level to logging.DEBUG (maximize logging output)
    Equivalent to: [--Application.log\_level=10]
--show-config
    Show the application's configuration (human-readable format)
    Equivalent to: [--Application.show\_config=True]
--show-config-json
    Show the application's configuration (json format)
    Equivalent to: [--Application.show\_config\_json=True]
--generate-config
    generate default config file
    Equivalent to: [--JupyterApp.generate\_config=True]
-y
    Answer yes to any questions instead of prompting.
    Equivalent to: [--JupyterApp.answer\_yes=True]
--execute
    Execute the notebook prior to export.
    Equivalent to: [--ExecutePreprocessor.enabled=True]
--allow-errors
    Continue notebook execution even if one of the cells throws an error and
include the error message in the cell output (the default behaviour is to abort
conversion). This flag is only relevant if '--execute' was specified, too.
    Equivalent to: [--ExecutePreprocessor.allow\_errors=True]
--stdin
    read a single notebook file from stdin. Write the resulting notebook with
default basename 'notebook.*'
    Equivalent to: [--NbConvertApp.from\_stdin=True]
--stdout
    Write notebook output to stdout instead of files.
    Equivalent to: [--NbConvertApp.writer\_class=StdoutWriter]
--inplace
    Run nbconvert in place, overwriting the existing notebook (only
            relevant when converting to notebook format)
    Equivalent to: [--NbConvertApp.use\_output\_suffix=False
--NbConvertApp.export\_format=notebook --FilesWriter.build\_directory=]
--clear-output
    Clear output of current file and save in place,
            overwriting the existing notebook.
    Equivalent to: [--NbConvertApp.use\_output\_suffix=False
--NbConvertApp.export\_format=notebook --FilesWriter.build\_directory=
--ClearOutputPreprocessor.enabled=True]
--coalesce-streams
    Coalesce consecutive stdout and stderr outputs into one stream (within each
cell).
    Equivalent to: [--NbConvertApp.use\_output\_suffix=False
--NbConvertApp.export\_format=notebook --FilesWriter.build\_directory=
--CoalesceStreamsPreprocessor.enabled=True]
--no-prompt
    Exclude input and output prompts from converted document.
    Equivalent to: [--TemplateExporter.exclude\_input\_prompt=True
--TemplateExporter.exclude\_output\_prompt=True]
--no-input
    Exclude input cells and output prompts from converted document.
            This mode is ideal for generating code-free reports.
    Equivalent to: [--TemplateExporter.exclude\_output\_prompt=True
--TemplateExporter.exclude\_input=True
--TemplateExporter.exclude\_input\_prompt=True]
--allow-chromium-download
    Whether to allow downloading chromium if no suitable version is found on the
system.
    Equivalent to: [--WebPDFExporter.allow\_chromium\_download=True]
--disable-chromium-sandbox
    Disable chromium security sandbox when converting to PDF..
    Equivalent to: [--WebPDFExporter.disable\_sandbox=True]
--show-input
    Shows code input. This flag is only useful for dejavu users.
    Equivalent to: [--TemplateExporter.exclude\_input=False]
--embed-images
    Embed the images as base64 dataurls in the output. This flag is only useful
for the HTML/WebPDF/Slides exports.
    Equivalent to: [--HTMLExporter.embed\_images=True]
--sanitize-html
    Whether the HTML in Markdown cells and cell outputs should be sanitized..
    Equivalent to: [--HTMLExporter.sanitize\_html=True]
--log-level=<Enum>
    Set the log level by value or name.
    Choices: any of [0, 10, 20, 30, 40, 50, 'DEBUG', 'INFO', 'WARN', 'ERROR',
'CRITICAL']
    Default: 30
    Equivalent to: [--Application.log\_level]
--config=<Unicode>
    Full path of a config file.
    Default: ''
    Equivalent to: [--JupyterApp.config\_file]
--to=<Unicode>
    The export format to be used, either one of the built-in formats
            ['asciidoc', 'custom', 'html', 'latex', 'markdown', 'notebook',
'pdf', 'python', 'qtpdf', 'qtpng', 'rst', 'script', 'slides', 'webpdf']
            or a dotted object name that represents the import path for an
            ``Exporter`` class
    Default: ''
    Equivalent to: [--NbConvertApp.export\_format]
--template=<Unicode>
    Name of the template to use
    Default: ''
    Equivalent to: [--TemplateExporter.template\_name]
--template-file=<Unicode>
    Name of the template file to use
    Default: None
    Equivalent to: [--TemplateExporter.template\_file]
--theme=<Unicode>
    Template specific theme(e.g. the name of a JupyterLab CSS theme distributed
    as prebuilt extension for the lab template)
    Default: 'light'
    Equivalent to: [--HTMLExporter.theme]
--sanitize\_html=<Bool>
    Whether the HTML in Markdown cells and cell outputs should be sanitized.This
    should be set to True by nbviewer or similar tools.
    Default: False
    Equivalent to: [--HTMLExporter.sanitize\_html]
--writer=<DottedObjectName>
    Writer class used to write the
                                        results of the conversion
    Default: 'FilesWriter'
    Equivalent to: [--NbConvertApp.writer\_class]
--post=<DottedOrNone>
    PostProcessor class used to write the
                                        results of the conversion
    Default: ''
    Equivalent to: [--NbConvertApp.postprocessor\_class]
--output=<Unicode>
    Overwrite base name use for output files.
                Supports pattern replacements '\{notebook\_name\}'.
    Default: '\{notebook\_name\}'
    Equivalent to: [--NbConvertApp.output\_base]
--output-dir=<Unicode>
    Directory to write output(s) to. Defaults
                                  to output to the directory of each notebook.
To recover
                                  previous default behaviour (outputting to the
current
                                  working directory) use . as the flag value.
    Default: ''
    Equivalent to: [--FilesWriter.build\_directory]
--reveal-prefix=<Unicode>
    The URL prefix for reveal.js (version 3.x).
            This defaults to the reveal CDN, but can be any url pointing to a
copy
            of reveal.js.
            For speaker notes to work, this must be a relative path to a local
            copy of reveal.js: e.g., "reveal.js".
            If a relative path is given, it must be a subdirectory of the
            current directory (from which the server is run).
            See the usage documentation
            (https://nbconvert.readthedocs.io/en/latest/usage.html\#reveal-js-
html-slideshow)
            for more details.
    Default: ''
    Equivalent to: [--SlidesExporter.reveal\_url\_prefix]
--nbformat=<Enum>
    The nbformat version to write.
            Use this to downgrade notebooks.
    Choices: any of [1, 2, 3, 4]
    Default: 4
    Equivalent to: [--NotebookExporter.nbformat\_version]

Examples
--------

    The simplest way to use nbconvert is

            > jupyter nbconvert mynotebook.ipynb --to html

            Options include ['asciidoc', 'custom', 'html', 'latex', 'markdown',
'notebook', 'pdf', 'python', 'qtpdf', 'qtpng', 'rst', 'script', 'slides',
'webpdf'].

            > jupyter nbconvert --to latex mynotebook.ipynb

            Both HTML and LaTeX support multiple output templates. LaTeX
includes
            'base', 'article' and 'report'.  HTML includes 'basic', 'lab' and
            'classic'. You can specify the flavor of the format used.

            > jupyter nbconvert --to html --template lab mynotebook.ipynb

            You can also pipe the output to stdout, rather than a file

            > jupyter nbconvert mynotebook.ipynb --stdout

            PDF is generated via latex

            > jupyter nbconvert mynotebook.ipynb --to pdf

            You can get (and serve) a Reveal.js-powered slideshow

            > jupyter nbconvert myslides.ipynb --to slides --post serve

            Multiple notebooks can be given at the command line in a couple of
            different ways:

            > jupyter nbconvert notebook*.ipynb
            > jupyter nbconvert notebook1.ipynb notebook2.ipynb

            or you can specify the notebooks list in a config file, containing::

                c.NbConvertApp.notebooks = ["my\_notebook.ipynb"]

            > jupyter nbconvert --config mycfg.py

To see all available configurables, use `--help-all`.

    \end{Verbatim}

    \begin{tcolorbox}[breakable, size=fbox, boxrule=1pt, pad at break*=1mm,colback=cellbackground, colframe=cellborder]
\prompt{In}{incolor}{23}{\boxspacing}
\begin{Verbatim}[commandchars=\\\{\}]
\PY{o}{!}apt\PYZhy{}get\PY{+w}{ }install\PY{+w}{ }\PYZhy{}y\PY{+w}{ }texlive\PYZhy{}xetex\PY{+w}{ }texlive\PYZhy{}fonts\PYZhy{}recommended\PY{+w}{ }texlive\PYZhy{}latex\PYZhy{}extra
\end{Verbatim}
\end{tcolorbox}

    \begin{Verbatim}[commandchars=\\\{\}]
Reading package lists{\ldots} Done
Building dependency tree{\ldots} Done
Reading state information{\ldots} Done
The following additional packages will be installed:
  dvisvgm fonts-droid-fallback fonts-lato fonts-lmodern fonts-noto-mono
  fonts-texgyre fonts-urw-base35 libapache-pom-java libcommons-logging-java
  libcommons-parent-java libfontbox-java libgs9 libgs9-common libidn12
  libijs-0.35 libjbig2dec0 libkpathsea6 libpdfbox-java libptexenc1 libruby3.0
  libsynctex2 libteckit0 libtexlua53 libtexluajit2 libwoff1 libzzip-0-13
  lmodern poppler-data preview-latex-style rake ruby ruby-net-telnet
  ruby-rubygems ruby-webrick ruby-xmlrpc ruby3.0 rubygems-integration t1utils
  teckit tex-common tex-gyre texlive-base texlive-binaries texlive-latex-base
  texlive-latex-recommended texlive-pictures texlive-plain-generic tipa
  xfonts-encodings xfonts-utils
Suggested packages:
  fonts-noto fonts-freefont-otf | fonts-freefont-ttf libavalon-framework-java
  libcommons-logging-java-doc libexcalibur-logkit-java liblog4j1.2-java
  poppler-utils ghostscript fonts-japanese-mincho | fonts-ipafont-mincho
  fonts-japanese-gothic | fonts-ipafont-gothic fonts-arphic-ukai
  fonts-arphic-uming fonts-nanum ri ruby-dev bundler debhelper gv
  | postscript-viewer perl-tk xpdf | pdf-viewer xzdec
  texlive-fonts-recommended-doc texlive-latex-base-doc python3-pygments
  icc-profiles libfile-which-perl libspreadsheet-parseexcel-perl
  texlive-latex-extra-doc texlive-latex-recommended-doc texlive-luatex
  texlive-pstricks dot2tex prerex texlive-pictures-doc vprerex
  default-jre-headless tipa-doc
The following NEW packages will be installed:
  dvisvgm fonts-droid-fallback fonts-lato fonts-lmodern fonts-noto-mono
  fonts-texgyre fonts-urw-base35 libapache-pom-java libcommons-logging-java
  libcommons-parent-java libfontbox-java libgs9 libgs9-common libidn12
  libijs-0.35 libjbig2dec0 libkpathsea6 libpdfbox-java libptexenc1 libruby3.0
  libsynctex2 libteckit0 libtexlua53 libtexluajit2 libwoff1 libzzip-0-13
  lmodern poppler-data preview-latex-style rake ruby ruby-net-telnet
  ruby-rubygems ruby-webrick ruby-xmlrpc ruby3.0 rubygems-integration t1utils
  teckit tex-common tex-gyre texlive-base texlive-binaries
  texlive-fonts-recommended texlive-latex-base texlive-latex-extra
  texlive-latex-recommended texlive-pictures texlive-plain-generic
  texlive-xetex tipa xfonts-encodings xfonts-utils
0 upgraded, 53 newly installed, 0 to remove and 35 not upgraded.
Need to get 182 MB of archives.
After this operation, 571 MB of additional disk space will be used.
Get:1 http://archive.ubuntu.com/ubuntu jammy/main amd64 fonts-droid-fallback all
1:6.0.1r16-1.1build1 [1,805 kB]
Get:2 http://archive.ubuntu.com/ubuntu jammy/main amd64 fonts-lato all 2.0-2.1
[2,696 kB]
Get:3 http://archive.ubuntu.com/ubuntu jammy/main amd64 poppler-data all
0.4.11-1 [2,171 kB]
Get:4 http://archive.ubuntu.com/ubuntu jammy/universe amd64 tex-common all 6.17
[33.7 kB]
Get:5 http://archive.ubuntu.com/ubuntu jammy/main amd64 fonts-urw-base35 all
20200910-1 [6,367 kB]
Get:6 http://archive.ubuntu.com/ubuntu jammy-updates/main amd64 libgs9-common
all 9.55.0\textasciitilde{}dfsg1-0ubuntu5.12 [753 kB]
Get:7 http://archive.ubuntu.com/ubuntu jammy-updates/main amd64 libidn12 amd64
1.38-4ubuntu1 [60.0 kB]
Get:8 http://archive.ubuntu.com/ubuntu jammy/main amd64 libijs-0.35 amd64
0.35-15build2 [16.5 kB]
Get:9 http://archive.ubuntu.com/ubuntu jammy/main amd64 libjbig2dec0 amd64
0.19-3build2 [64.7 kB]
Get:10 http://archive.ubuntu.com/ubuntu jammy-updates/main amd64 libgs9 amd64
9.55.0\textasciitilde{}dfsg1-0ubuntu5.12 [5,031 kB]
Get:11 http://archive.ubuntu.com/ubuntu jammy-updates/main amd64 libkpathsea6
amd64 2021.20210626.59705-1ubuntu0.2 [60.4 kB]
Get:12 http://archive.ubuntu.com/ubuntu jammy/main amd64 libwoff1 amd64
1.0.2-1build4 [45.2 kB]
Get:13 http://archive.ubuntu.com/ubuntu jammy/universe amd64 dvisvgm amd64
2.13.1-1 [1,221 kB]
Get:14 http://archive.ubuntu.com/ubuntu jammy/universe amd64 fonts-lmodern all
2.004.5-6.1 [4,532 kB]
Get:15 http://archive.ubuntu.com/ubuntu jammy/main amd64 fonts-noto-mono all
20201225-1build1 [397 kB]
Get:16 http://archive.ubuntu.com/ubuntu jammy/universe amd64 fonts-texgyre all
20180621-3.1 [10.2 MB]
Get:17 http://archive.ubuntu.com/ubuntu jammy/universe amd64 libapache-pom-java
all 18-1 [4,720 B]
Get:18 http://archive.ubuntu.com/ubuntu jammy/universe amd64 libcommons-parent-
java all 43-1 [10.8 kB]
Get:19 http://archive.ubuntu.com/ubuntu jammy/universe amd64 libcommons-logging-
java all 1.2-2 [60.3 kB]
Get:20 http://archive.ubuntu.com/ubuntu jammy-updates/main amd64 libptexenc1
amd64 2021.20210626.59705-1ubuntu0.2 [39.1 kB]
Get:21 http://archive.ubuntu.com/ubuntu jammy/main amd64 rubygems-integration
all 1.18 [5,336 B]
Get:22 http://archive.ubuntu.com/ubuntu jammy-updates/main amd64 ruby3.0 amd64
3.0.2-7ubuntu2.10 [50.1 kB]
Get:23 http://archive.ubuntu.com/ubuntu jammy/main amd64 ruby-rubygems all
3.3.5-2 [228 kB]
Get:24 http://archive.ubuntu.com/ubuntu jammy/main amd64 ruby amd64 1:3.0\textasciitilde{}exp1
[5,100 B]
Get:25 http://archive.ubuntu.com/ubuntu jammy/main amd64 rake all 13.0.6-2 [61.7
kB]
Get:26 http://archive.ubuntu.com/ubuntu jammy/main amd64 ruby-net-telnet all
0.1.1-2 [12.6 kB]
Ign:27 http://archive.ubuntu.com/ubuntu jammy-updates/main amd64 ruby-webrick
all 1.7.0-3ubuntu0.1
Get:28 http://archive.ubuntu.com/ubuntu jammy-updates/main amd64 ruby-xmlrpc all
0.3.2-1ubuntu0.1 [24.9 kB]
Get:29 http://archive.ubuntu.com/ubuntu jammy-updates/main amd64 libruby3.0
amd64 3.0.2-7ubuntu2.10 [5,114 kB]
Get:30 http://archive.ubuntu.com/ubuntu jammy-updates/main amd64 libsynctex2
amd64 2021.20210626.59705-1ubuntu0.2 [55.6 kB]
Get:31 http://archive.ubuntu.com/ubuntu jammy/universe amd64 libteckit0 amd64
2.5.11+ds1-1 [421 kB]
Get:32 http://archive.ubuntu.com/ubuntu jammy-updates/main amd64 libtexlua53
amd64 2021.20210626.59705-1ubuntu0.2 [120 kB]
Get:33 http://archive.ubuntu.com/ubuntu jammy-updates/main amd64 libtexluajit2
amd64 2021.20210626.59705-1ubuntu0.2 [267 kB]
Get:34 http://archive.ubuntu.com/ubuntu jammy/universe amd64 libzzip-0-13 amd64
0.13.72+dfsg.1-1.1 [27.0 kB]
Get:35 http://archive.ubuntu.com/ubuntu jammy/main amd64 xfonts-encodings all
1:1.0.5-0ubuntu2 [578 kB]
Get:36 http://archive.ubuntu.com/ubuntu jammy/main amd64 xfonts-utils amd64
1:7.7+6build2 [94.6 kB]
Err:27 http://security.ubuntu.com/ubuntu jammy-updates/main amd64 ruby-webrick
all 1.7.0-3ubuntu0.1
  404  Not Found [IP: 91.189.91.83 80]
Get:37 http://archive.ubuntu.com/ubuntu jammy/universe amd64 lmodern all
2.004.5-6.1 [9,471 kB]
Get:38 http://archive.ubuntu.com/ubuntu jammy/universe amd64 preview-latex-style
all 12.2-1ubuntu1 [185 kB]
Get:39 http://archive.ubuntu.com/ubuntu jammy/main amd64 t1utils amd64
1.41-4build2 [61.3 kB]
Get:40 http://archive.ubuntu.com/ubuntu jammy/universe amd64 teckit amd64
2.5.11+ds1-1 [699 kB]
Get:41 http://archive.ubuntu.com/ubuntu jammy/universe amd64 tex-gyre all
20180621-3.1 [6,209 kB]
Get:42 http://archive.ubuntu.com/ubuntu jammy-updates/universe amd64 texlive-
binaries amd64 2021.20210626.59705-1ubuntu0.2 [9,860 kB]
Get:43 http://archive.ubuntu.com/ubuntu jammy/universe amd64 texlive-base all
2021.20220204-1 [21.0 MB]
Get:44 http://archive.ubuntu.com/ubuntu jammy/universe amd64 texlive-fonts-
recommended all 2021.20220204-1 [4,972 kB]
Get:45 http://archive.ubuntu.com/ubuntu jammy/universe amd64 texlive-latex-base
all 2021.20220204-1 [1,128 kB]
Get:46 http://archive.ubuntu.com/ubuntu jammy/universe amd64 libfontbox-java all
1:1.8.16-2 [207 kB]
Get:47 http://archive.ubuntu.com/ubuntu jammy/universe amd64 libpdfbox-java all
1:1.8.16-2 [5,199 kB]
Get:48 http://archive.ubuntu.com/ubuntu jammy/universe amd64 texlive-latex-
recommended all 2021.20220204-1 [14.4 MB]
Get:49 http://archive.ubuntu.com/ubuntu jammy/universe amd64 texlive-pictures
all 2021.20220204-1 [8,720 kB]
Get:50 http://archive.ubuntu.com/ubuntu jammy/universe amd64 texlive-latex-extra
all 2021.20220204-1 [13.9 MB]
Get:51 http://archive.ubuntu.com/ubuntu jammy/universe amd64 texlive-plain-
generic all 2021.20220204-1 [27.5 MB]
Get:52 http://archive.ubuntu.com/ubuntu jammy/universe amd64 tipa all 2:1.3-21
[2,967 kB]
Get:53 http://archive.ubuntu.com/ubuntu jammy/universe amd64 texlive-xetex all
2021.20220204-1 [12.4 MB]
Fetched 181 MB in 13s (13.6 MB/s)
E: Failed to fetch http://security.ubuntu.com/ubuntu/pool/main/r/ruby-
webrick/ruby-webrick\_1.7.0-3ubuntu0.1\_all.deb  404  Not Found [IP: 91.189.91.83
80]
E: Unable to fetch some archives, maybe run apt-get update or try with --fix-
missing?
    \end{Verbatim}


    % Add a bibliography block to the postdoc
    
    
    
\end{document}
